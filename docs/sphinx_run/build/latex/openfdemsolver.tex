%% Generated by Sphinx.
\def\sphinxdocclass{report}
\documentclass[letterpaper,10pt,english]{sphinxmanual}
\ifdefined\pdfpxdimen
   \let\sphinxpxdimen\pdfpxdimen\else\newdimen\sphinxpxdimen
\fi \sphinxpxdimen=.75bp\relax

\PassOptionsToPackage{warn}{textcomp}
\usepackage[utf8]{inputenc}
\ifdefined\DeclareUnicodeCharacter
% support both utf8 and utf8x syntaxes
  \ifdefined\DeclareUnicodeCharacterAsOptional
    \def\sphinxDUC#1{\DeclareUnicodeCharacter{"#1}}
  \else
    \let\sphinxDUC\DeclareUnicodeCharacter
  \fi
  \sphinxDUC{00A0}{\nobreakspace}
  \sphinxDUC{2500}{\sphinxunichar{2500}}
  \sphinxDUC{2502}{\sphinxunichar{2502}}
  \sphinxDUC{2514}{\sphinxunichar{2514}}
  \sphinxDUC{251C}{\sphinxunichar{251C}}
  \sphinxDUC{2572}{\textbackslash}
\fi
\usepackage{cmap}
\usepackage[T1]{fontenc}
\usepackage{amsmath,amssymb,amstext}
\usepackage{babel}



\usepackage{times}
\expandafter\ifx\csname T@LGR\endcsname\relax
\else
% LGR was declared as font encoding
  \substitutefont{LGR}{\rmdefault}{cmr}
  \substitutefont{LGR}{\sfdefault}{cmss}
  \substitutefont{LGR}{\ttdefault}{cmtt}
\fi
\expandafter\ifx\csname T@X2\endcsname\relax
  \expandafter\ifx\csname T@T2A\endcsname\relax
  \else
  % T2A was declared as font encoding
    \substitutefont{T2A}{\rmdefault}{cmr}
    \substitutefont{T2A}{\sfdefault}{cmss}
    \substitutefont{T2A}{\ttdefault}{cmtt}
  \fi
\else
% X2 was declared as font encoding
  \substitutefont{X2}{\rmdefault}{cmr}
  \substitutefont{X2}{\sfdefault}{cmss}
  \substitutefont{X2}{\ttdefault}{cmtt}
\fi


\usepackage[Bjarne]{fncychap}
\usepackage{sphinx}

\fvset{fontsize=\small}
\usepackage{geometry}


% Include hyperref last.
\usepackage{hyperref}
% Fix anchor placement for figures with captions.
\usepackage{hypcap}% it must be loaded after hyperref.
% Set up styles of URL: it should be placed after hyperref.
\urlstyle{same}

\addto\captionsenglish{\renewcommand{\contentsname}{Contents:}}

\usepackage{sphinxmessages}
\setcounter{tocdepth}{1}



\title{OpenFDEM Solver}
\date{Jan 02, 2023}
\release{0.0}
\author{Grasselli\textquotesingle{}s Geomechanics Group}
\newcommand{\sphinxlogo}{\vbox{}}
\renewcommand{\releasename}{Release}
\makeindex
\begin{document}

\pagestyle{empty}
\sphinxmaketitle
\pagestyle{plain}
\sphinxtableofcontents
\pagestyle{normal}
\phantomsection\label{\detokenize{index::doc}}



\chapter{Features and About OpenFDEM}
\label{\detokenize{rst_about_project/intro:features-and-about-openfdem}}\label{\detokenize{rst_about_project/intro::doc}}
OpenFDEM aims to be a free finite and discrete element solver with object\sphinxhyphen{}oriented architecture for solving
\sphinxstylestrong{multiscale, multiphase, and multiphysics} (3M) problems. The applications are,
but not limited to, mechanical, thermal and fluid mechanics problems. The main features are:

\sphinxstylestrong{Modular \& Extensible FEM Kernel and DEM Kernel} (OPENFDEMlib)
\begin{itemize}
\item {} 
\sphinxstylestrong{Fully extendable and portable} \sphinxhyphen{} The kernel is extendable in any “direction”. The possibility of adding new element types, new material models with any element type and number of internal history parameters, new boundary conditions (time\sphinxhyphen{}dependent, position\sphinxhyphen{}dependent, state\sphinxhyphen{}dependent, periodic and flow\sphinxhyphen{}in/out) or numerical algorithms (explicit and implicit), as well as ability to add and manage arbitrary degrees of freedom.

\item {} 
\sphinxstylestrong{Flexible element definition} \sphinxhyphen{} OpenFDEM is not limited only to triangular elements, but allows Q4 cohesive element and triangle\sphinxhyphen{}triangle contact. OpenFDEM is a more general FEM/DEM solver can be compatible to arbitrary scenarios.

\item {} 
\sphinxstylestrong{Highly accurate and reliable} \textendash{} The kernel provides high\sphinxhyphen{}order integration schemes and solving methods to seek more reliable numerical results which are comparable to theoretical solutions. The element type has a maximum order of three to rebuild the large deformation within the entity, and the kinematic is able to construct a nonlinear deformation. The Hilber\sphinxhyphen{}Hughes\sphinxhyphen{}Taylor (HHT) time integration scheme (second order accuracy) is employed for the explicit solver.

\item {} 
\sphinxstylestrong{Friendly preprocessing interface} \sphinxhyphen{} The Gmsh API is provided to quickly create meshes from CAD, geometry file and third\sphinxhyphen{}party commercial software. The built\sphinxhyphen{}in commands are accessible to crate many basic geometries (rectangular, circle, ellipse, polygon, line, particles) and initial discontinuities (single joint, joint sets, Discrete Fracture Networks (DFNs), DFNs from image mapping). The built\sphinxhyphen{}in mesh module is able to quickly assess mesh quality and the local bad meshes will be further optimized by swap, node insertion, node delete, element split techniques, automatically or manually.

\item {} 
\sphinxstylestrong{Parallel processing support} \sphinxhyphen{} Most modules can be operated in parallel and very good performance scalability can be obtained across various operating platforms.

\item {} 
\sphinxstylestrong{Mesh adaptive analysis support} \textendash{} Local adaptive mesh refinement (LAMR) and global adaptive mesh refinement (GAMR) are provided for mesh optimization and accuracy enhancement. It supports various error estimations based on different remeshing criteria, support for primary unknown and internal variables mapping, support for high\sphinxhyphen{}accuracy internal variable interpolation and fast unbalance equilibrium after refinement. The AMR supports fracture mapping before and after remeshing.

\item {} 
\sphinxstylestrong{Rich grain\sphinxhyphen{}based modelling support} \textendash{} Voronoi tessellations can be created with the built\sphinxhyphen{}in Voronoi module. The optimization is deployed to match the laboratorial mineral distribution from measurements or digital image. The realistic GBM can be reproduced directly in the project by inputting the binary sample images, the polygonal element type is available for representing the whole mineral individually, further transgranular fracturing can be realized by element splitting techniques.

\item {} 
\sphinxstylestrong{Large material library including the state\sphinxhyphen{}of\sphinxhyphen{}the\sphinxhyphen{}art models for phase field of quasibrittle materials and rich element library} \textendash{} currently, OpenFDEM supports 17 element materials spanning elastic, hyperelastic, plastic, damage, nonlocal, viscous and phasefiled models, supports 7 cohesive materials spanning static, dynamic and fatigue problems. It also supports 6 contact models including Mohr\sphinxhyphen{}Coulomb friction, Hertz contact, rate friction, and rough dilation shear law.

\item {} 
\sphinxstylestrong{Advanced Analysis Solvers} \sphinxhyphen{} Linear dynamic (implicit and explicit), linear static (using PETSC) and nonlinear dynamic (explicit) are applicable for different problems.

\end{itemize}

\sphinxstylestrong{Particle Discrete Element Method (pDEM)}
\begin{itemize}
\item {} 
\sphinxstylestrong{Rigid DEM support} \sphinxhyphen{} Built in module for rigid particles packing, kinematics and collision, the particle\sphinxhyphen{}based contact models include linear, Hertz, cohesive bond and rotation resistance model.

\item {} 
\sphinxstylestrong{Realistic Particle Modelling} \sphinxhyphen{} Overlapping particles and Fourier\sphinxhyphen{}Voronoi\sphinxhyphen{}based algorithm are used to generate realistic particles having complex shapes. The realistic particles can be rigid or deformable, the breakage of the particles are also possible.

\end{itemize}

\sphinxstylestrong{Fluid Dynamic Module}
\begin{itemize}
\item {} 
\sphinxstylestrong{Analysis Procedures} \sphinxhyphen{}  Matrix flow for pore seepage, transient incompressible fracture flow, transient compressible fracture flow and gas flow problems.

\item {} 
\sphinxstylestrong{Element Library} \sphinxhyphen{} Triangle, quadratic triangle, quadrilateral and quadratic quadrilateral element types are supported for Newtonian fluid and Bingham fluid.

\item {} 
\sphinxstylestrong{Boundary Types} \sphinxhyphen{} Water level, pore pressure, flow rate, steady flow and impermeable boundary conditions are supported in hydro module.

\end{itemize}

\sphinxstylestrong{Thermal Transportation Module}
\begin{itemize}
\item {} 
\sphinxstylestrong{Analysis procedures} \sphinxhyphen{} matrix thermal transportation, thermal resistance in fractures, heat conduction of fluid in fracture, heat advection of fluid, heat exchange between solid and fluid and contact thermal problems.

\item {} 
\sphinxstylestrong{Element Library} \sphinxhyphen{} triangle, quadratic triangle, quadrilateral and quadratic quadrilateral element types are supported.

\item {} 
\sphinxstylestrong{Boundary Types} \sphinxhyphen{} constant temperature, flux, conduction, advection, radiation, source and adiabatic thermal conditions are supported.

\end{itemize}

\sphinxstylestrong{Computational Fluid Dynamics}
\begin{itemize}
\item {} 
\sphinxstylestrong{Material Point Method (MPM)} is used to simulate the fluid transportation and large deformation. This mesh\sphinxhyphen{}free method does not encounter the drawbacks of mesh\sphinxhyphen{}based methods (high deformation tangling, advection errors etc.) which makes it a promising and powerful tool for large deformation problems. The coupling among FDEM and MPM makes the solid interacting with fluid is possible.

\end{itemize}

\sphinxstylestrong{Post\sphinxhyphen{}Processing}
\begin{itemize}
\item {} 
Export to VTK format is supported, allowing to use VTK based visualization tools (such as ParaView) for postprocessing on different operating platforms.

\item {} 
Export to Tecplot format is supported.

\item {} 
Export historic variables which are monitored at each step to csv is supported.

\end{itemize}

\sphinxstylestrong{Third\sphinxhyphen{}Party Packages Used in OpenFDEM}
\begin{itemize}
\item {} 
\sphinxcode{\sphinxupquote{GMSH}} \sphinxhyphen{} 2D and 3D mesh generator

\item {} 
\sphinxcode{\sphinxupquote{GSL}} \sphinxhyphen{} mathematical routines

\item {} 
\sphinxcode{\sphinxupquote{Eigen}} \sphinxhyphen{} matrix calculation

\item {} 
\sphinxcode{\sphinxupquote{PETSC}} \sphinxhyphen{} Portable, Extensible Toolkit for Scientific Computation

\item {} 
\sphinxcode{\sphinxupquote{ParaView}} \sphinxhyphen{} Parallel Visualization Application (for .vtk files)

\end{itemize}


\section{Documentation}
\label{\detokenize{rst_about_project/intro:documentation}}
The documentation is auto\sphinxhyphen{}generated from the \sphinxcode{\sphinxupquote{.of}} and \sphinxcode{\sphinxupquote{.rst}} files throughout
the codebase and the extensive comments in the source code \sphinxcode{\sphinxupquote{.h}} and \sphinxcode{\sphinxupquote{.of}} files. Sphinx is used to compile
the documentation in HTML and PDF formats.


\chapter{System Requirements}
\label{\detokenize{rst_about_project/requirements:system-requirements}}\label{\detokenize{rst_about_project/requirements::doc}}

\section{Build Requirements}
\label{\detokenize{rst_about_project/requirements:build-requirements}}
To compile \sphinxcode{\sphinxupquote{OpenFDEM}}, you neeed a complier supporting \sphinxcode{\sphinxupquote{C++ 17}} and the following packages are required:
\begin{itemize}
\item {} 
\sphinxstylestrong{CMake} (\textgreater{}=3.5.1)

\item {} 
\sphinxstylestrong{OpenMP} \sphinxhyphen{} a high\sphinxhyphen{}performance, freely available package for multi core acceleration

\item {} 
\sphinxstylestrong{Gmsh} \sphinxhyphen{}  mesh generation and pre\sphinxhyphen{}processing, it is optional and the kernel is implemented in the source code (4.10)

\item {} 
\sphinxstylestrong{Eigen} \sphinxhyphen{} a scientific matrix computation, it is optional and the headers are included in source code (\textgreater{}=3.4.0)

\end{itemize}


\section{Post\sphinxhyphen{}Processing}
\label{\detokenize{rst_about_project/requirements:post-processing}}
To use the post\sphinxhyphen{}processing outputs (optional steps):
\begin{itemize}
\item {} 
\sphinxstylestrong{ParaView} \sphinxhyphen{} Parallel visualization application

\item {} 
\sphinxstylestrong{Tecplot} \sphinxhyphen{} Commerical software for field results

\end{itemize}


\section{Implicit Static/Non\sphinxhyphen{}Linear Solvers}
\label{\detokenize{rst_about_project/requirements:implicit-static-non-linear-solvers}}
To use the implicit static or nonlinear solvers, at least one of the following libraries is required:
\begin{itemize}
\item {} 
\sphinxstylestrong{PETSc} \sphinxhyphen{} portable, extensible toolkit for scientific computation

\item {} 
\sphinxstylestrong{LAPACK} \sphinxhyphen{} a standard software library for numerical linear algebra

\end{itemize}

\sphinxcode{\sphinxupquote{OpenFDEM}} is flexible and can be run on Windows or Linux\sphinxhyphen{}like systems. The released version is for Windows x64.


\chapter{Quick Start for Developers}
\label{\detokenize{rst_about_project/started:quick-start-for-developers}}\label{\detokenize{rst_about_project/started::doc}}
This is the source code of C/C++ based OpenFDEM project developed by Dr. Xiaofeng Li,
since 2017. This project is not limited to an FDEM solver for continuum\sphinxhyphen{}discontinuum problems,
but is also capable of solving particulate DEM, material point method and phasefield problems.

The project can be complied by Visual Studio (\textgreater{} 2015) and compatible with Windows 7, 10 or Linux\sphinxhyphen{}like systems.

Tutorial examples can be found in \sphinxcode{\sphinxupquote{..src\textbackslash{}test\textbackslash{}..}}. The main file is located at \sphinxcode{\sphinxupquote{src\textbackslash{}solve\textbackslash{}openfdem.cpp}}. OpenFDEM is run by
parsing the input file.

General steps to run OpenFDEM models:
\begin{enumerate}
\sphinxsetlistlabels{\arabic}{enumi}{enumii}{}{.}%
\item {} 
Open \sphinxcode{\sphinxupquote{.sln}} project in \sphinxcode{\sphinxupquote{\textbackslash{}openfdem src\textbackslash{}of\textbackslash{}OpenFDEM\textbackslash{}OpenFDEM.sln}} by your local Visual Studio software.

\item {} 
Compile the project in Visual Studio  from the \sphinxcode{\sphinxupquote{src\textbackslash{}solve\textbackslash{}openfdem.cpp}} main file.

\item {} 
The executable code should be in \sphinxcode{\sphinxupquote{x64\textbackslash{}Release\textasciigrave{}\textasciigrave{}(or \textasciigrave{}\textasciigrave{}Debug}}), be sure to keep the \sphinxcode{\sphinxupquote{.dll}} files in the same folder with \sphinxcode{\sphinxupquote{.exe}} file.

\item {} 
Drag the \sphinxcode{\sphinxupquote{.of}} file into the \sphinxcode{\sphinxupquote{.exe}} software then the model will starts to run, or use \sphinxcode{\sphinxupquote{openfdem example.of}} in terminal to run a model.

\end{enumerate}


\section{Source Code Download}
\label{\detokenize{rst_about_project/started:source-code-download}}
The source code is hosted on the University of Toronto Gitlab: \sphinxhref{http://geogroup.utoronto.ca:9191/gitlab/xiaofeng.li/openfdem\_solver}{OpenFDEM Gitlab}.


\section{System Requirements}
\label{\detokenize{rst_about_project/started:system-requirements}}\begin{itemize}
\item {} 
Windows x64

\item {} 
Visual Studio (\textgreater{} 2015)

\end{itemize}

All external dependencies are included in the OpenFDEM download.


\chapter{Copyright}
\label{\detokenize{rst_about_project/copyrights:copyright}}\label{\detokenize{rst_about_project/copyrights::doc}}\begin{sphinxalltt}
              \_\_\_\_                   \_\_\_\_\_\_ \_\_\_\_\_  \_\_\_\_\_\_ \_\_  \_\_                     
             / \_\_ \textbackslash{}                 |  \_\_\_\_|  \_\_ \textbackslash{}|  \_\_\_\_|  \textbackslash{}/  |                    
            | |  | |\_ \_\_   \_\_\_ \_ \_\_ | |\_\_  | |  | | |\_\_  | \textbackslash{}  / |                    
            | |  | | \textquotesingle{}\_ \textbackslash{} / \_ \textbackslash{} \textquotesingle{}\_ \textbackslash{}|  \_\_| | |  | |  \_\_| | |\textbackslash{}/| |                    
            | |\_\_| | |\_) |  \_\_/ | | | |    | |\_\_| | |\_\_\_\_| |  | |                    
             \textbackslash{}\_\_\_\_/| .\_\_/ \textbackslash{}\_\_\_|\_| |\_|\_|    |\_\_\_\_\_/|\_\_\_\_\_\_|\_|  |\_|                    
                   | |                     OpenFree Finite Element                   
                   |\_|                         and Discrete Element Method Solver    
                                                                                     
        OpenFDEM : Object Oriented Open Free Finite Discrete Element Code   
                                                                                     

            Copyright (C) 2017 \sphinxhyphen{} 2022   Xiaofeng Li                                  
                         Email: xfli@whrsm.ac.cn                                     
                                                                                     
     OpenFDEM Project Website: https://openfdem.com/
         Geomechanics Group Website: https://geogroup.utoronto.ca/
         
     This library is free software; you can redistribute it and/or                   
     modify it under the terms of the GNU Lesser General Public                      
     License as published by the Free Software Foundation; either                    
     version 2.1 of the License, or any later version.              
                                                                                     
     This program is distributed in the hope that it will be useful,                 
     but WITHOUT ANY WARRANTY; without even the implied warranty of                  
     MERCHANTABILITY or FITNESS FOR A PARTICULAR PURPOSE.  See the GNU               
     Lesser General Public License for more details.  
\end{sphinxalltt}


\chapter{Tutorial 1: UCS model from Gmsh}
\label{\detokenize{rst_tutorials/tutorial1_ucs:tutorial-1-ucs-model-from-gmsh}}\label{\detokenize{rst_tutorials/tutorial1_ucs::doc}}
This example will review how to setup a uniaxial compressive test example with output from Gmsh software.

\sphinxstylestrong{Runtime}: \textless{}10 min on i9 8\sphinxhyphen{}core Windows 10 Machine

Expected tutorial output (visualized in ParaView):

\noindent{\hspace*{\fill}\sphinxincludegraphics[width=800\sphinxpxdimen]{{ucs_example}.png}\hspace*{\fill}}

OpenFDEM supports four main mesh pre\sphinxhyphen{}processing approaches:
\begin{enumerate}
\sphinxsetlistlabels{\arabic}{enumi}{enumii}{}{.}%
\item {} 
A user defined command in OpenFDEM to create mesh automatically

\item {} 
Importing a \sphinxcode{\sphinxupquote{.geo}} file

\item {} 
Importing a \sphinxcode{\sphinxupquote{.msh}} file

\item {} 
Mesh developed from other commercial softwares, including \sphinxcode{\sphinxupquote{.inp}}, \sphinxcode{\sphinxupquote{.dxf}}, \sphinxcode{\sphinxupquote{.fdem}}, \sphinxcode{\sphinxupquote{.tess}} (for grain\sphinxhyphen{}based model only) and \sphinxcode{\sphinxupquote{.jpg}} (for grain\sphinxhyphen{}based module and DFN module only).

\end{enumerate}


\section{Tutorial Prerequistes}
\label{\detokenize{rst_tutorials/tutorial1_ucs:tutorial-prerequistes}}
The following files are needed to follow along the tutorial:
\begin{itemize}
\item {} 
\sphinxhref{http://geogroup.utoronto.ca:9191/gitlab/xiaofeng.li/openfdem\_solver/-/blob/main/openfdem\%20src/src/test/UCS\%20by\%20gmsh/example\_UCS.geo}{example\_UCS.geo} (click to download from Gitlab)

\item {} 
\sphinxhref{http://geogroup.utoronto.ca:9191/gitlab/xiaofeng.li/openfdem\_solver/-/blob/main/openfdem\%20src/src/test/UCS\%20by\%20gmsh/example\_UCS.msh}{example\_UCS.msh} (click to download from Gitlab)

\end{itemize}


\section{Tutorial Steps}
\label{\detokenize{rst_tutorials/tutorial1_ucs:tutorial-steps}}
OpenFDEM tutorials have the same main steps:
\begin{enumerate}
\sphinxsetlistlabels{\arabic}{enumi}{enumii}{}{.}%
\item {} 
Mesh pre\sphinxhyphen{}processing steps.

\item {} 
Materials definition.

\item {} 
Define boundary conditions.

\item {} 
Specify the outputs.

\end{enumerate}

A \sphinxcode{\sphinxupquote{.log}} file is auto\sphinxhyphen{}generated by default in the same folder as the input file. To disable the logging, specify this
line into the input file:

\begin{sphinxVerbatim}[commandchars=\\\{\}]
\PYG{n}{of}\PYG{o}{.}\PYG{n}{debug} \PYG{n}{off}
\end{sphinxVerbatim}


\subsection{Mesh Pre\sphinxhyphen{}processing}
\label{\detokenize{rst_tutorials/tutorial1_ucs:mesh-pre-processing}}
Create a new empty text file (later add the \sphinxtitleref{.of} extension). Begin writing the following commands:

\begin{sphinxVerbatim}[commandchars=\\\{\}]
\PYG{n}{of}\PYG{o}{.}\PYG{n}{new} 
\end{sphinxVerbatim}

\begin{sphinxVerbatim}[commandchars=\\\{\}]
\PYG{c+c1}{\PYGZsh{} create a domain xmin =\PYGZhy{}25e\PYGZhy{}3, xmax =25e\PYGZhy{}3 , ymin=\PYGZhy{}50e\PYGZhy{}3, ymax=50e\PYGZhy{}3 }
\PYG{c+c1}{\PYGZsh{} The domain is mandatory when the particles or material points are used }
\PYG{n}{of}\PYG{o}{.}\PYG{n}{geometry}\PYG{o}{.}\PYG{n}{domain} \PYG{o}{\PYGZhy{}}\PYG{l+m+mf}{25e\PYGZhy{}3} \PYG{l+m+mf}{25e\PYGZhy{}3}  \PYG{o}{\PYGZhy{}}\PYG{l+m+mf}{50e\PYGZhy{}3}  \PYG{l+m+mf}{50e\PYGZhy{}3} 
\end{sphinxVerbatim}

\begin{sphinxVerbatim}[commandchars=\\\{\}]
\PYG{c+c1}{\PYGZsh{} Create a retangular block, group tag is specimen, the range is xmin =25e\PYGZhy{}3, xmax =25e\PYGZhy{}3, ymin=\PYGZhy{}50e\PYGZhy{}3,ymax= 50e\PYGZhy{}3}
\PYG{n}{of}\PYG{o}{.}\PYG{n}{geometry}\PYG{o}{.}\PYG{n}{square} \PYG{l+s+s1}{\PYGZsq{}}\PYG{l+s+s1}{specimen}\PYG{l+s+s1}{\PYGZsq{}} \PYG{o}{\PYGZhy{}}\PYG{l+m+mf}{25e\PYGZhy{}3} \PYG{l+m+mf}{25e\PYGZhy{}3}  \PYG{o}{\PYGZhy{}}\PYG{l+m+mf}{50e\PYGZhy{}3}  \PYG{l+m+mf}{50e\PYGZhy{}3} 
\end{sphinxVerbatim}

\begin{sphinxVerbatim}[commandchars=\\\{\}]
\PYG{c+c1}{\PYGZsh{} Create a block for the upper plate}
\PYG{n}{of}\PYG{o}{.}\PYG{n}{geometry}\PYG{o}{.}\PYG{n}{square} \PYG{l+s+s1}{\PYGZsq{}}\PYG{l+s+s1}{up\PYGZus{}plate}\PYG{l+s+s1}{\PYGZsq{}} \PYG{o}{\PYGZhy{}}\PYG{l+m+mf}{35e\PYGZhy{}3} \PYG{l+m+mf}{35e\PYGZhy{}3}  \PYG{l+m+mf}{50e\PYGZhy{}3}  \PYG{l+m+mf}{55e\PYGZhy{}3} 
\end{sphinxVerbatim}

\begin{sphinxVerbatim}[commandchars=\\\{\}]
\PYG{c+c1}{\PYGZsh{} Create a block for the down plate}
\PYG{n}{of}\PYG{o}{.}\PYG{n}{geometry}\PYG{o}{.}\PYG{n}{square} \PYG{l+s+s1}{\PYGZsq{}}\PYG{l+s+s1}{down\PYGZus{}plate}\PYG{l+s+s1}{\PYGZsq{}} \PYG{o}{\PYGZhy{}}\PYG{l+m+mf}{35e\PYGZhy{}3} \PYG{l+m+mf}{35e\PYGZhy{}3}  \PYG{o}{\PYGZhy{}}\PYG{l+m+mf}{55e\PYGZhy{}3}  \PYG{o}{\PYGZhy{}}\PYG{l+m+mf}{50e\PYGZhy{}3} 
\end{sphinxVerbatim}

After creating geometry entities, you can assign mesh size to the whole model with \sphinxcode{\sphinxupquote{all}} keyword and it is also possible to assign
mesh size to specific entities:

\begin{sphinxVerbatim}[commandchars=\\\{\}]
\PYG{l+s+sd}{\PYGZsq{}\PYGZsq{}\PYGZsq{} There are four methods to create mesh, 1\PYGZhy{} user defined commands in prepprocessing, it will call gmsh kernel to mesh the geometry}
\PYG{l+s+sd}{, 2\PYGZhy{} import .geo file for gmsh, 3\PYGZhy{} import .msh file (less than V 2.4) and 4\PYGZhy{} import .inp file from other codes \PYGZsq{}\PYGZsq{}\PYGZsq{}}
\PYG{c+c1}{\PYGZsh{} assign global mesh size, the default keyword is for global entities}
\PYG{n}{of}\PYG{o}{.}\PYG{n}{geometry}\PYG{o}{.}\PYG{n}{mesh}\PYG{o}{.}\PYG{n}{size} \PYG{l+s+s1}{\PYGZsq{}}\PYG{l+s+s1}{all}\PYG{l+s+s1}{\PYGZsq{}} \PYG{l+m+mf}{10e\PYGZhy{}3}
\PYG{c+c1}{\PYGZsh{} assign specific mesh size to group \PYGZsq{}specimen\PYGZsq{}}
\PYG{n}{of}\PYG{o}{.}\PYG{n}{geometry}\PYG{o}{.}\PYG{n}{mesh}\PYG{o}{.}\PYG{n}{size} \PYG{l+s+s1}{\PYGZsq{}}\PYG{l+s+s1}{specimen}\PYG{l+s+s1}{\PYGZsq{}} \PYG{l+m+mf}{5e\PYGZhy{}3}
\end{sphinxVerbatim}

\sphinxcode{\sphinxupquote{OpenFDEM}} will call the Gmsh kernel to generate mesh after the mesh size is assigned. The meshing scheme includes \sphinxcode{\sphinxupquote{delaunay}} (default),
\sphinxcode{\sphinxupquote{meshadapt}} and \sphinxcode{\sphinxupquote{frontal\sphinxhyphen{}delaunay}} . The gmsh interface will be called and you can check the mesh quality, recombine the mesh or change mesh size in gmsh pannel.

\begin{sphinxVerbatim}[commandchars=\\\{\}]
\PYG{c+c1}{\PYGZsh{} starts to mesh, delaunay is optional, it is the default value}
\PYG{n}{of}\PYG{o}{.}\PYG{n}{geometry}\PYG{o}{.}\PYG{n}{mesh} \PYG{n}{delaunay}
\end{sphinxVerbatim}

Cohesive elements can be inserted after importing the mesh. OpenFDEM supports to partially insert cohesive elements and also supports inserting extrinsic cohesive elements.

\begin{sphinxVerbatim}[commandchars=\\\{\}]
\PYG{c+c1}{\PYGZsh{}insert cohesive elements, it is aviable to insert CZM in the whole model or to a specific entity}
\PYG{n}{of}\PYG{o}{.}\PYG{n}{mesh}\PYG{o}{.}\PYG{n}{insert} \PYG{l+s+s1}{\PYGZsq{}}\PYG{l+s+s1}{specimen}\PYG{l+s+s1}{\PYGZsq{}}
\end{sphinxVerbatim}


\subsection{Materials Definition}
\label{\detokenize{rst_tutorials/tutorial1_ucs:materials-definition}}
Material parameters contain three parts:
\sphinxhyphen{} The parameter for solid matrix
\sphinxhyphen{} Solid cohesive elements and
\sphinxhyphen{} Contacts

The paramters will be allocated to the user\sphinxhyphen{}defined element groups.

\begin{sphinxVerbatim}[commandchars=\\\{\}]
\PYG{c+c1}{\PYGZsh{} assign  material parameters to solid elements based on the element groups}
\PYG{n}{of}\PYG{o}{.}\PYG{n}{mat}\PYG{o}{.}\PYG{n}{element} \PYG{l+s+s1}{\PYGZsq{}}\PYG{l+s+s1}{specimen}\PYG{l+s+s1}{\PYGZsq{}} \PYG{n}{ELASTIC} \PYG{n}{den} \PYG{l+m+mi}{2700} \PYG{n}{E} \PYG{l+m+mf}{30e9} \PYG{n}{v} \PYG{l+m+mf}{0.3} \PYG{n}{damp} \PYG{l+m+mf}{0.6}
\PYG{n}{of}\PYG{o}{.}\PYG{n}{mat}\PYG{o}{.}\PYG{n}{element} \PYG{l+s+s1}{\PYGZsq{}}\PYG{l+s+s1}{up\PYGZus{}plate}\PYG{l+s+s1}{\PYGZsq{}} \PYG{n}{ELASTIC} \PYG{n}{den} \PYG{l+m+mi}{2700}  \PYG{n}{E} \PYG{l+m+mf}{70e9} \PYG{n}{v} \PYG{l+m+mf}{0.2} \PYG{n}{damp} \PYG{l+m+mf}{0.9}
\PYG{n}{of}\PYG{o}{.}\PYG{n}{mat}\PYG{o}{.}\PYG{n}{element} \PYG{l+s+s1}{\PYGZsq{}}\PYG{l+s+s1}{down\PYGZus{}plate}\PYG{l+s+s1}{\PYGZsq{}} \PYG{n}{ELASTIC} \PYG{n}{den} \PYG{l+m+mi}{2700} \PYG{n}{E} \PYG{l+m+mf}{70e9} \PYG{n}{v} \PYG{l+m+mf}{0.2} \PYG{n}{damp} \PYG{l+m+mf}{0.9}

\PYG{c+c1}{\PYGZsh{}assign material parameters to cohesive elements, the default is a reserved keyword means the whole entities}
\PYG{c+c1}{\PYGZsh{} user are not allowed to use this keyword for the group tags}
\PYG{n}{of}\PYG{o}{.}\PYG{n}{mat}\PYG{o}{.}\PYG{n}{cohesive} \PYG{l+s+s1}{\PYGZsq{}}\PYG{l+s+s1}{all}\PYG{l+s+s1}{\PYGZsq{}} \PYG{n}{EM} \PYG{n}{ten} \PYG{l+m+mf}{1e6} \PYG{n}{coh} \PYG{l+m+mf}{3e6} \PYG{n}{fric} \PYG{l+m+mf}{0.3} \PYG{n}{GI} \PYG{l+m+mi}{10} \PYG{n}{GII} \PYG{l+m+mi}{50} \PYG{n}{beta\PYGZus{}I} \PYG{l+m+mi}{0} \PYG{n}{beta\PYGZus{}II} \PYG{l+m+mi}{0}

\PYG{c+c1}{\PYGZsh{}assign material parameters to contact}
\PYG{n}{of}\PYG{o}{.}\PYG{n}{mat}\PYG{o}{.}\PYG{n}{contact} \PYG{l+s+s1}{\PYGZsq{}}\PYG{l+s+s1}{all}\PYG{l+s+s1}{\PYGZsq{}} \PYG{n}{MC} \PYG{n}{fric} \PYG{l+m+mf}{0.3}
\end{sphinxVerbatim}


\subsection{Define Boundary Conditions}
\label{\detokenize{rst_tutorials/tutorial1_ucs:define-boundary-conditions}}
The boundary conditions are defined by the entity groups.

\begin{sphinxVerbatim}[commandchars=\\\{\}]
\PYG{c+c1}{\PYGZsh{}create up\PYGZus{}plate and up\PYGZus{}plate nodal physical groups from element groups}
\PYG{c+c1}{\PYGZsh{} the element physcial groups are inherited from the msh file in geometry}
\PYG{n}{of}\PYG{o}{.}\PYG{n}{group}\PYG{o}{.}\PYG{n}{nodal}\PYG{o}{.}\PYG{n}{from}\PYG{o}{.}\PYG{n}{element} \PYG{l+s+s1}{\PYGZsq{}}\PYG{l+s+s1}{up\PYGZus{}plate}\PYG{l+s+s1}{\PYGZsq{}} \PYG{l+s+s1}{\PYGZsq{}}\PYG{l+s+s1}{up\PYGZus{}plate}\PYG{l+s+s1}{\PYGZsq{}}
\PYG{n}{of}\PYG{o}{.}\PYG{n}{group}\PYG{o}{.}\PYG{n}{nodal}\PYG{o}{.}\PYG{n}{from}\PYG{o}{.}\PYG{n}{element} \PYG{l+s+s1}{\PYGZsq{}}\PYG{l+s+s1}{down\PYGZus{}plate}\PYG{l+s+s1}{\PYGZsq{}} \PYG{l+s+s1}{\PYGZsq{}}\PYG{l+s+s1}{down\PYGZus{}plate}\PYG{l+s+s1}{\PYGZsq{}}

\PYG{c+c1}{\PYGZsh{}assign Dirichlet boundaries, the fixed velocities in X and y directions are added to the }
\PYG{c+c1}{\PYGZsh{} nodal groups }
\PYG{n}{of}\PYG{o}{.}\PYG{n}{boundary}\PYG{o}{.}\PYG{n}{nodal}\PYG{o}{.}\PYG{n}{velocity}  \PYG{l+s+s1}{\PYGZsq{}}\PYG{l+s+s1}{up\PYGZus{}plate}\PYG{l+s+s1}{\PYGZsq{}} \PYG{n}{XY} \PYG{l+m+mf}{0.0} \PYG{o}{\PYGZhy{}}\PYG{l+m+mf}{0.05}
\PYG{n}{of}\PYG{o}{.}\PYG{n}{boundary}\PYG{o}{.}\PYG{n}{nodal}\PYG{o}{.}\PYG{n}{velocity}  \PYG{l+s+s1}{\PYGZsq{}}\PYG{l+s+s1}{down\PYGZus{}plate}\PYG{l+s+s1}{\PYGZsq{}} \PYG{n}{XY} \PYG{l+m+mf}{0.0} \PYG{l+m+mf}{0.05}
\end{sphinxVerbatim}


\subsection{Post\sphinxhyphen{}Processing Settings}
\label{\detokenize{rst_tutorials/tutorial1_ucs:post-processing-settings}}
\begin{sphinxVerbatim}[commandchars=\\\{\}]
\PYG{c+c1}{\PYGZsh{}set for post\PYGZhy{}processing; how often to output the results + variables (of.history.all would export all variables, but large file....)}
\PYG{c+c1}{\PYGZsh{} interval to write history}
\PYG{n}{of}\PYG{o}{.}\PYG{n}{history}\PYG{o}{.}\PYG{n}{interval} \PYG{l+m+mi}{10}
\PYG{c+c1}{\PYGZsh{} interval to write paraview field results}
\PYG{n}{of}\PYG{o}{.}\PYG{n}{history}\PYG{o}{.}\PYG{n}{pv}\PYG{o}{.}\PYG{n}{interval} \PYG{l+m+mi}{5000}
\PYG{l+s+sd}{\PYGZsq{}\PYGZsq{}\PYGZsq{}Export field variables, default is  to export all variables, the user can choose to export specific variables using keywords\PYGZsq{}\PYGZsq{}\PYGZsq{}}
\PYG{n}{of}\PYG{o}{.}\PYG{n}{history}\PYG{o}{.}\PYG{n}{pv}\PYG{o}{.}\PYG{n}{field} \PYG{n+nb}{all}
\PYG{n}{of}\PYG{o}{.}\PYG{n}{history}\PYG{o}{.}\PYG{n}{pv}\PYG{o}{.}\PYG{n}{fracture} \PYG{n+nb}{all}

\PYG{c+c1}{\PYGZsh{} monitor the average nodal displacemnt in upper plate every step}
\PYG{n}{of}\PYG{o}{.}\PYG{n}{history}\PYG{o}{.}\PYG{n}{nodal}\PYG{o}{.}\PYG{n}{group}\PYG{o}{.}\PYG{n}{dis} \PYG{l+m+mi}{1} \PYG{l+s+s1}{\PYGZsq{}}\PYG{l+s+s1}{up\PYGZus{}plate}\PYG{l+s+s1}{\PYGZsq{}}
\PYG{c+c1}{\PYGZsh{} monitor the average stress (tensor) in the specimen}
\PYG{n}{of}\PYG{o}{.}\PYG{n}{history}\PYG{o}{.}\PYG{n}{element}\PYG{o}{.}\PYG{n}{group}\PYG{o}{.}\PYG{n}{stress} \PYG{l+m+mi}{2} \PYG{l+s+s1}{\PYGZsq{}}\PYG{l+s+s1}{specimen}\PYG{l+s+s1}{\PYGZsq{}}

\end{sphinxVerbatim}


\subsection{Run models}
\label{\detokenize{rst_tutorials/tutorial1_ucs:run-models}}
Finally, define the number of time\sphinxhyphen{}steps:

\begin{sphinxVerbatim}[commandchars=\\\{\}]
\PYG{c+c1}{\PYGZsh{} total run steps}
\PYG{n}{of}\PYG{o}{.}\PYG{n}{step} \PYG{l+m+mi}{150000}
\end{sphinxVerbatim}


\subsection{Full Tutorial Script}
\label{\detokenize{rst_tutorials/tutorial1_ucs:full-tutorial-script}}
To run the model, save your text file with the \sphinxtitleref{.of} extension. Rebuild the openfdem solution and drag your
\sphinxtitleref{.of} file into the \sphinxtitleref{OpenFDEM.exe}. It will automatically run and save the outputs.

Complete script below:

\begin{sphinxVerbatim}[commandchars=\\\{\}]
\PYG{n}{of}\PYG{o}{.}\PYG{n}{geometry}\PYG{o}{.}\PYG{n}{square} \PYG{l+s+s1}{\PYGZsq{}}\PYG{l+s+s1}{specimen}\PYG{l+s+s1}{\PYGZsq{}} \PYG{o}{\PYGZhy{}}\PYG{l+m+mf}{25e\PYGZhy{}3} \PYG{l+m+mf}{25e\PYGZhy{}3}  \PYG{o}{\PYGZhy{}}\PYG{l+m+mf}{50e\PYGZhy{}3}  \PYG{l+m+mf}{50e\PYGZhy{}3} 
\PYG{c+c1}{\PYGZsh{} Create a block for the upper plate}
\PYG{n}{of}\PYG{o}{.}\PYG{n}{geometry}\PYG{o}{.}\PYG{n}{square} \PYG{l+s+s1}{\PYGZsq{}}\PYG{l+s+s1}{up\PYGZus{}plate}\PYG{l+s+s1}{\PYGZsq{}} \PYG{o}{\PYGZhy{}}\PYG{l+m+mf}{35e\PYGZhy{}3} \PYG{l+m+mf}{35e\PYGZhy{}3}  \PYG{l+m+mf}{50e\PYGZhy{}3}  \PYG{l+m+mf}{55e\PYGZhy{}3} 
\PYG{c+c1}{\PYGZsh{} Create a block for the down plate}
\PYG{n}{of}\PYG{o}{.}\PYG{n}{geometry}\PYG{o}{.}\PYG{n}{square} \PYG{l+s+s1}{\PYGZsq{}}\PYG{l+s+s1}{down\PYGZus{}plate}\PYG{l+s+s1}{\PYGZsq{}} \PYG{o}{\PYGZhy{}}\PYG{l+m+mf}{35e\PYGZhy{}3} \PYG{l+m+mf}{35e\PYGZhy{}3}  \PYG{o}{\PYGZhy{}}\PYG{l+m+mf}{55e\PYGZhy{}3}  \PYG{o}{\PYGZhy{}}\PYG{l+m+mf}{50e\PYGZhy{}3} 

\PYG{l+s+sd}{\PYGZsq{}\PYGZsq{}\PYGZsq{} There are four methods to create mesh, 1\PYGZhy{} user defined commands in prepprocessing, it will call gmsh kernel to mesh the geometry}
\PYG{l+s+sd}{, 2\PYGZhy{} import .geo file for gmsh, 3\PYGZhy{} import .msh file (less than V 2.4) and 4\PYGZhy{} import .inp file from other codes \PYGZsq{}\PYGZsq{}\PYGZsq{}}
\PYG{c+c1}{\PYGZsh{} assign global mesh size, the default keyword is for global entities}
\PYG{n}{of}\PYG{o}{.}\PYG{n}{geometry}\PYG{o}{.}\PYG{n}{mesh}\PYG{o}{.}\PYG{n}{size} \PYG{l+s+s1}{\PYGZsq{}}\PYG{l+s+s1}{all}\PYG{l+s+s1}{\PYGZsq{}} \PYG{l+m+mf}{10e\PYGZhy{}3}
\PYG{c+c1}{\PYGZsh{} assign specific mesh size to group \PYGZsq{}specimen\PYGZsq{}}
\PYG{n}{of}\PYG{o}{.}\PYG{n}{geometry}\PYG{o}{.}\PYG{n}{mesh}\PYG{o}{.}\PYG{n}{size} \PYG{l+s+s1}{\PYGZsq{}}\PYG{l+s+s1}{specimen}\PYG{l+s+s1}{\PYGZsq{}} \PYG{l+m+mf}{5e\PYGZhy{}3}

\PYG{c+c1}{\PYGZsh{} starts to mesh, delaunay is optional, it is the default value}
\PYG{n}{of}\PYG{o}{.}\PYG{n}{geometry}\PYG{o}{.}\PYG{n}{mesh} \PYG{n}{delaunay}

\PYG{c+c1}{\PYGZsh{}insert cohesive elements, it is aviable to insert CZM in the whole model or to a specific entity}
\PYG{n}{of}\PYG{o}{.}\PYG{n}{mesh}\PYG{o}{.}\PYG{n}{insert} \PYG{l+s+s1}{\PYGZsq{}}\PYG{l+s+s1}{specimen}\PYG{l+s+s1}{\PYGZsq{}}

\PYG{c+c1}{\PYGZsh{} assign  material parameters to solid elements based on the element groups}
\PYG{n}{of}\PYG{o}{.}\PYG{n}{mat}\PYG{o}{.}\PYG{n}{element} \PYG{l+s+s1}{\PYGZsq{}}\PYG{l+s+s1}{specimen}\PYG{l+s+s1}{\PYGZsq{}} \PYG{n}{ELASTIC} \PYG{n}{den} \PYG{l+m+mi}{2700} \PYG{n}{E} \PYG{l+m+mf}{30e9} \PYG{n}{v} \PYG{l+m+mf}{0.3} \PYG{n}{damp} \PYG{l+m+mf}{0.6}
\PYG{n}{of}\PYG{o}{.}\PYG{n}{mat}\PYG{o}{.}\PYG{n}{element} \PYG{l+s+s1}{\PYGZsq{}}\PYG{l+s+s1}{up\PYGZus{}plate}\PYG{l+s+s1}{\PYGZsq{}} \PYG{n}{ELASTIC} \PYG{n}{den} \PYG{l+m+mi}{2700}  \PYG{n}{E} \PYG{l+m+mf}{70e9} \PYG{n}{v} \PYG{l+m+mf}{0.2} \PYG{n}{damp} \PYG{l+m+mf}{0.9}
\PYG{n}{of}\PYG{o}{.}\PYG{n}{mat}\PYG{o}{.}\PYG{n}{element} \PYG{l+s+s1}{\PYGZsq{}}\PYG{l+s+s1}{down\PYGZus{}plate}\PYG{l+s+s1}{\PYGZsq{}} \PYG{n}{ELASTIC} \PYG{n}{den} \PYG{l+m+mi}{2700} \PYG{n}{E} \PYG{l+m+mf}{70e9} \PYG{n}{v} \PYG{l+m+mf}{0.2} \PYG{n}{damp} \PYG{l+m+mf}{0.9}

\PYG{c+c1}{\PYGZsh{}assign material parameters to cohesive elements, the default is a reserved keyword means the whole entities}
\PYG{c+c1}{\PYGZsh{} user are not allowed to use this keyword for the group tags}
\PYG{n}{of}\PYG{o}{.}\PYG{n}{mat}\PYG{o}{.}\PYG{n}{cohesive} \PYG{l+s+s1}{\PYGZsq{}}\PYG{l+s+s1}{all}\PYG{l+s+s1}{\PYGZsq{}} \PYG{n}{EM} \PYG{n}{ten} \PYG{l+m+mf}{1e6} \PYG{n}{coh} \PYG{l+m+mf}{3e6} \PYG{n}{fric} \PYG{l+m+mf}{0.3} \PYG{n}{GI} \PYG{l+m+mi}{10} \PYG{n}{GII} \PYG{l+m+mi}{50} \PYG{n}{beta\PYGZus{}I} \PYG{l+m+mi}{0} \PYG{n}{beta\PYGZus{}II} \PYG{l+m+mi}{0}

\PYG{c+c1}{\PYGZsh{}assign material parameters to contact}
\PYG{n}{of}\PYG{o}{.}\PYG{n}{mat}\PYG{o}{.}\PYG{n}{contact} \PYG{l+s+s1}{\PYGZsq{}}\PYG{l+s+s1}{all}\PYG{l+s+s1}{\PYGZsq{}} \PYG{n}{MC} \PYG{n}{fric} \PYG{l+m+mf}{0.3}

\PYG{c+c1}{\PYGZsh{}create up\PYGZus{}plate and up\PYGZus{}plate nodal physical groups from element groups}
\PYG{c+c1}{\PYGZsh{} the element physcial groups are inherited from the msh file in geometry}
\PYG{n}{of}\PYG{o}{.}\PYG{n}{group}\PYG{o}{.}\PYG{n}{nodal}\PYG{o}{.}\PYG{n}{from}\PYG{o}{.}\PYG{n}{element} \PYG{l+s+s1}{\PYGZsq{}}\PYG{l+s+s1}{up\PYGZus{}plate}\PYG{l+s+s1}{\PYGZsq{}} \PYG{l+s+s1}{\PYGZsq{}}\PYG{l+s+s1}{up\PYGZus{}plate}\PYG{l+s+s1}{\PYGZsq{}}
\PYG{n}{of}\PYG{o}{.}\PYG{n}{group}\PYG{o}{.}\PYG{n}{nodal}\PYG{o}{.}\PYG{n}{from}\PYG{o}{.}\PYG{n}{element} \PYG{l+s+s1}{\PYGZsq{}}\PYG{l+s+s1}{down\PYGZus{}plate}\PYG{l+s+s1}{\PYGZsq{}} \PYG{l+s+s1}{\PYGZsq{}}\PYG{l+s+s1}{down\PYGZus{}plate}\PYG{l+s+s1}{\PYGZsq{}}

\PYG{c+c1}{\PYGZsh{}assign Dirichlet boundaries, the fixed velocities in X and y directions are added to the }
\PYG{c+c1}{\PYGZsh{} nodal groups }
\PYG{n}{of}\PYG{o}{.}\PYG{n}{boundary}\PYG{o}{.}\PYG{n}{nodal}\PYG{o}{.}\PYG{n}{velocity}  \PYG{l+s+s1}{\PYGZsq{}}\PYG{l+s+s1}{up\PYGZus{}plate}\PYG{l+s+s1}{\PYGZsq{}} \PYG{n}{XY} \PYG{l+m+mf}{0.0} \PYG{o}{\PYGZhy{}}\PYG{l+m+mf}{0.05}
\PYG{n}{of}\PYG{o}{.}\PYG{n}{boundary}\PYG{o}{.}\PYG{n}{nodal}\PYG{o}{.}\PYG{n}{velocity}  \PYG{l+s+s1}{\PYGZsq{}}\PYG{l+s+s1}{down\PYGZus{}plate}\PYG{l+s+s1}{\PYGZsq{}} \PYG{n}{XY} \PYG{l+m+mf}{0.0} \PYG{l+m+mf}{0.05}

\PYG{c+c1}{\PYGZsh{}assign global damping value, the value should not be over 1.0}
\PYG{c+c1}{\PYGZsh{} the default value is 0.7}
\PYG{n}{of}\PYG{o}{.}\PYG{n}{damp}\PYG{o}{.}\PYG{k}{global} \PYG{l+m+mf}{0.5}

\PYG{c+c1}{\PYGZsh{}set for post\PYGZhy{}processing; how often to output the results + variables (of.history.all would export all variables, but large file....)}
\PYG{c+c1}{\PYGZsh{} interval to write history}
\PYG{n}{of}\PYG{o}{.}\PYG{n}{history}\PYG{o}{.}\PYG{n}{interval} \PYG{l+m+mi}{10}
\PYG{c+c1}{\PYGZsh{} interval to write paraview field results}
\PYG{n}{of}\PYG{o}{.}\PYG{n}{history}\PYG{o}{.}\PYG{n}{pv}\PYG{o}{.}\PYG{n}{interval} \PYG{l+m+mi}{5000}
\PYG{l+s+sd}{\PYGZsq{}\PYGZsq{}\PYGZsq{}Export field variables, default is  to export all variables, the user can choose to export specific variables using keywords\PYGZsq{}\PYGZsq{}\PYGZsq{}}
\PYG{n}{of}\PYG{o}{.}\PYG{n}{history}\PYG{o}{.}\PYG{n}{pv}\PYG{o}{.}\PYG{n}{field} \PYG{n+nb}{all}
\PYG{n}{of}\PYG{o}{.}\PYG{n}{history}\PYG{o}{.}\PYG{n}{pv}\PYG{o}{.}\PYG{n}{fracture} \PYG{n+nb}{all}

\PYG{c+c1}{\PYGZsh{} monitor the average nodal displacemnt in upper plate every step}
\PYG{n}{of}\PYG{o}{.}\PYG{n}{history}\PYG{o}{.}\PYG{n}{nodal}\PYG{o}{.}\PYG{n}{group}\PYG{o}{.}\PYG{n}{dis} \PYG{l+m+mi}{1} \PYG{l+s+s1}{\PYGZsq{}}\PYG{l+s+s1}{up\PYGZus{}plate}\PYG{l+s+s1}{\PYGZsq{}}
\PYG{c+c1}{\PYGZsh{} monitor the average stress (tensor) in the specimen}
\PYG{n}{of}\PYG{o}{.}\PYG{n}{history}\PYG{o}{.}\PYG{n}{element}\PYG{o}{.}\PYG{n}{group}\PYG{o}{.}\PYG{n}{stress} \PYG{l+m+mi}{2} \PYG{l+s+s1}{\PYGZsq{}}\PYG{l+s+s1}{specimen}\PYG{l+s+s1}{\PYGZsq{}}

\PYG{c+c1}{\PYGZsh{} total run steps}
\PYG{n}{of}\PYG{o}{.}\PYG{n}{step} \PYG{l+m+mi}{150000}
\end{sphinxVerbatim}


\chapter{Tutorial 2: In Situ Stress Tutorial}
\label{\detokenize{rst_tutorials/tutorial2_insitu:tutorial-2-in-situ-stress-tutorial}}\label{\detokenize{rst_tutorials/tutorial2_insitu::doc}}
This example will review how to setup an insitu excavation example, defining the mesh in the \sphinxstyleemphasis{.of} file.

\sphinxstylestrong{Runtime}: \textless{}10 min on i9 8\sphinxhyphen{}core Windows 10 Machine

Expected tutorial output (visualized in ParaView):

\noindent{\hspace*{\fill}\sphinxincludegraphics[width=400\sphinxpxdimen]{{insitu_example}.png}\hspace*{\fill}}


\section{Tutorial Resources}
\label{\detokenize{rst_tutorials/tutorial2_insitu:tutorial-resources}}
The result mesh file will be created during the tutorial, but can also be viewed on Gitlab: \sphinxhref{http://geogroup.utoronto.ca:9191/gitlab/xiaofeng.li/openfdem\_solver/-/blob/main/openfdem\%20src/src/test/insitustress/mesh.msh}{mesh.msh}.


\section{Tutorial Steps}
\label{\detokenize{rst_tutorials/tutorial2_insitu:tutorial-steps}}

\subsection{Mesh Pre\sphinxhyphen{}Processing}
\label{\detokenize{rst_tutorials/tutorial2_insitu:mesh-pre-processing}}
In a new text\sphinxhyphen{}file, write the following commands. These will define the geometry
as a square with a circle in the center for excavation:

\begin{sphinxVerbatim}[commandchars=\\\{\}]
\PYG{c+c1}{\PYGZsh{} Create a retangular block, group tag is rock, the range is xmin =0, xmax =1, ymin=0,ymax= 1}
\PYG{n}{of}\PYG{o}{.}\PYG{n}{geometry}\PYG{o}{.}\PYG{n}{square} \PYG{l+s+s1}{\PYGZsq{}}\PYG{l+s+s1}{rock}\PYG{l+s+s1}{\PYGZsq{}} \PYG{l+m+mi}{0} \PYG{l+m+mi}{1} \PYG{l+m+mi}{0} \PYG{l+m+mi}{1}
\PYG{c+c1}{\PYGZsh{} Create a hole in the block, cut is to fragment the block and seprate the rock block to hole block and }
\PYG{c+c1}{\PYGZsh{} new rock block (out of the hole block)}
\PYG{n}{of}\PYG{o}{.}\PYG{n}{geometry}\PYG{o}{.}\PYG{n}{cut}\PYG{o}{.}\PYG{n}{circle} \PYG{l+s+s1}{\PYGZsq{}}\PYG{l+s+s1}{hole}\PYG{l+s+s1}{\PYGZsq{}} \PYG{l+s+s1}{\PYGZsq{}}\PYG{l+s+s1}{rock}\PYG{l+s+s1}{\PYGZsq{}} \PYG{l+m+mf}{0.5} \PYG{l+m+mf}{0.5} \PYG{l+m+mf}{0.1} \PYG{l+m+mi}{70}
\PYG{c+c1}{\PYGZsh{} assign global mesh size, the default keyword is for global entities}
\PYG{n}{of}\PYG{o}{.}\PYG{n}{geometry}\PYG{o}{.}\PYG{n}{mesh}\PYG{o}{.}\PYG{n}{size} \PYG{l+s+s1}{\PYGZsq{}}\PYG{l+s+s1}{all}\PYG{l+s+s1}{\PYGZsq{}} \PYG{l+m+mf}{0.02}
\PYG{c+c1}{\PYGZsh{} starts to mesh using auto method, delaunay the default value}
\PYG{n}{of}\PYG{o}{.}\PYG{n}{geometry}\PYG{o}{.}\PYG{n}{mesh} \PYG{n}{auto}
\end{sphinxVerbatim}

\begin{sphinxVerbatim}[commandchars=\\\{\}]
\PYG{c+c1}{\PYGZsh{}create excavation element group using cricle tool, based on the centric point and radius}
\PYG{n}{of}\PYG{o}{.}\PYG{n}{group}\PYG{o}{.}\PYG{n}{element}\PYG{o}{.}\PYG{n}{circle} \PYG{l+s+s1}{\PYGZsq{}}\PYG{l+s+s1}{excavation}\PYG{l+s+s1}{\PYGZsq{}} \PYG{l+m+mf}{0.5} \PYG{l+m+mf}{0.5} \PYG{l+m+mf}{0.1}
\end{sphinxVerbatim}


\subsection{Materials Definition}
\label{\detokenize{rst_tutorials/tutorial2_insitu:materials-definition}}
Assign the following material properties (density, Young’s modulus and damping coefficient):

\begin{sphinxVerbatim}[commandchars=\\\{\}]
\PYG{c+c1}{\PYGZsh{} assign  material parameters to solid elements based on the element groups}
\PYG{n}{of}\PYG{o}{.}\PYG{n}{mat}\PYG{o}{.}\PYG{n}{element} \PYG{l+s+s1}{\PYGZsq{}}\PYG{l+s+s1}{all}\PYG{l+s+s1}{\PYGZsq{}} \PYG{n}{ELASTIC} \PYG{n}{den} \PYG{l+m+mi}{2000} \PYG{n}{E} \PYG{l+m+mf}{30e9} \PYG{n}{v} \PYG{l+m+mf}{0.3} \PYG{n}{damp} \PYG{l+m+mf}{0.9}
\PYG{n}{of}\PYG{o}{.}\PYG{n}{mat}\PYG{o}{.}\PYG{n}{contact} \PYG{l+s+s1}{\PYGZsq{}}\PYG{l+s+s1}{all}\PYG{l+s+s1}{\PYGZsq{}} \PYG{n}{MC} \PYG{n}{fric} \PYG{l+m+mf}{0.3}
\end{sphinxVerbatim}


\subsection{Define Boundary Conditions}
\label{\detokenize{rst_tutorials/tutorial2_insitu:define-boundary-conditions}}
\begin{sphinxVerbatim}[commandchars=\\\{\}]
\PYG{c+c1}{\PYGZsh{} assign the in\PYGZhy{}situ stress in the domain}
\PYG{n}{of}\PYG{o}{.}\PYG{n}{boundary}\PYG{o}{.}\PYG{n}{element}\PYG{o}{.}\PYG{n}{stress} \PYG{o}{\PYGZhy{}}\PYG{l+m+mf}{35e6} \PYG{l+m+mf}{0.0} \PYG{o}{\PYGZhy{}}\PYG{l+m+mf}{35e6}

\PYG{c+c1}{\PYGZsh{} create nodal groups on edges}
\PYG{n}{of}\PYG{o}{.}\PYG{n}{group}\PYG{o}{.}\PYG{n}{nodal}\PYG{o}{.}\PYG{n}{plane} \PYG{l+s+s1}{\PYGZsq{}}\PYG{l+s+s1}{bottom\PYGZus{}edge}\PYG{l+s+s1}{\PYGZsq{}} \PYG{l+m+mf}{0.0} \PYG{l+m+mf}{0.0} \PYG{l+m+mf}{1.0} \PYG{l+m+mf}{0.0}
\PYG{n}{of}\PYG{o}{.}\PYG{n}{group}\PYG{o}{.}\PYG{n}{nodal}\PYG{o}{.}\PYG{n}{plane} \PYG{l+s+s1}{\PYGZsq{}}\PYG{l+s+s1}{up\PYGZus{}edge}\PYG{l+s+s1}{\PYGZsq{}} \PYG{l+m+mf}{0.0} \PYG{l+m+mf}{1.0} \PYG{l+m+mf}{1.0} \PYG{l+m+mf}{1.0}
\PYG{n}{of}\PYG{o}{.}\PYG{n}{group}\PYG{o}{.}\PYG{n}{nodal}\PYG{o}{.}\PYG{n}{plane} \PYG{l+s+s1}{\PYGZsq{}}\PYG{l+s+s1}{left\PYGZus{}edge}\PYG{l+s+s1}{\PYGZsq{}} \PYG{l+m+mf}{0.0} \PYG{l+m+mf}{0.0} \PYG{l+m+mf}{0.0} \PYG{l+m+mf}{1.0}
\PYG{n}{of}\PYG{o}{.}\PYG{n}{group}\PYG{o}{.}\PYG{n}{nodal}\PYG{o}{.}\PYG{n}{plane} \PYG{l+s+s1}{\PYGZsq{}}\PYG{l+s+s1}{right\PYGZus{}edge}\PYG{l+s+s1}{\PYGZsq{}} \PYG{l+m+mf}{1.0} \PYG{l+m+mf}{0.0} \PYG{l+m+mf}{1.0} \PYG{l+m+mf}{1.0}
\end{sphinxVerbatim}


\subsection{Run Model, Set Paraview Parameters}
\label{\detokenize{rst_tutorials/tutorial2_insitu:run-model-set-paraview-parameters}}
Define parameters for paraview export:

\begin{sphinxVerbatim}[commandchars=\\\{\}]
\PYG{c+c1}{\PYGZsh{} set interval to write paraview field results}
\PYG{n}{of}\PYG{o}{.}\PYG{n}{history}\PYG{o}{.}\PYG{n}{pv}\PYG{o}{.}\PYG{n}{interval} \PYG{l+m+mi}{500}
\PYG{n}{of}\PYG{o}{.}\PYG{n}{history}\PYG{o}{.}\PYG{n}{pv}\PYG{o}{.}\PYG{n}{field} \PYG{n+nb}{all}
\PYG{n}{of}\PYG{o}{.}\PYG{n}{history}\PYG{o}{.}\PYG{n}{pv}\PYG{o}{.}\PYG{n}{fracture} \PYG{n+nb}{all}
\PYG{n}{of}\PYG{o}{.}\PYG{n}{history}\PYG{o}{.}\PYG{n}{pv}\PYG{o}{.}\PYG{n}{cohesive} \PYG{n+nb}{all}

\PYG{c+c1}{\PYGZsh{} run steps to equlibrum the in\PYGZhy{}situ stress, it should be fast}
\PYG{n}{of}\PYG{o}{.}\PYG{n}{step} \PYG{l+m+mi}{1000}
\end{sphinxVerbatim}


\subsection{Set Tunnel Excavation}
\label{\detokenize{rst_tutorials/tutorial2_insitu:set-tunnel-excavation}}
\begin{sphinxVerbatim}[commandchars=\\\{\}]
\PYG{c+c1}{\PYGZsh{} excavte the tunnel}
\PYG{n}{of}\PYG{o}{.}\PYG{n}{boundary}\PYG{o}{.}\PYG{n}{excavation} \PYG{l+s+s1}{\PYGZsq{}}\PYG{l+s+s1}{excavation}\PYG{l+s+s1}{\PYGZsq{}}
\PYG{c+c1}{\PYGZsh{} fix the outer boundaries}
\PYG{n}{of}\PYG{o}{.}\PYG{n}{boundary}\PYG{o}{.}\PYG{n}{nodal}\PYG{o}{.}\PYG{n}{velocity} \PYG{l+s+s1}{\PYGZsq{}}\PYG{l+s+s1}{bottom}\PYG{l+s+s1}{\PYGZsq{}} \PYG{n}{Y} \PYG{l+m+mi}{0} 
\PYG{n}{of}\PYG{o}{.}\PYG{n}{boundary}\PYG{o}{.}\PYG{n}{nodal}\PYG{o}{.}\PYG{n}{velocity} \PYG{l+s+s1}{\PYGZsq{}}\PYG{l+s+s1}{up}\PYG{l+s+s1}{\PYGZsq{}} \PYG{n}{Y} \PYG{l+m+mi}{0} 
\PYG{n}{of}\PYG{o}{.}\PYG{n}{boundary}\PYG{o}{.}\PYG{n}{nodal}\PYG{o}{.}\PYG{n}{velocity} \PYG{l+s+s1}{\PYGZsq{}}\PYG{l+s+s1}{left}\PYG{l+s+s1}{\PYGZsq{}} \PYG{n}{X} \PYG{l+m+mi}{0} 
\PYG{n}{of}\PYG{o}{.}\PYG{n}{boundary}\PYG{o}{.}\PYG{n}{nodal}\PYG{o}{.}\PYG{n}{velocity} \PYG{l+s+s1}{\PYGZsq{}}\PYG{l+s+s1}{right}\PYG{l+s+s1}{\PYGZsq{}} \PYG{n}{X} \PYG{l+m+mi}{0} 
\end{sphinxVerbatim}


\subsection{Execute Model}
\label{\detokenize{rst_tutorials/tutorial2_insitu:execute-model}}
Define the number of model time\sphinxhyphen{}steps:

\begin{sphinxVerbatim}[commandchars=\\\{\}]
\PYG{n}{of}\PYG{o}{.}\PYG{n}{step} \PYG{l+m+mi}{50000}
\PYG{c+c1}{\PYGZsh{} terminate the run and step out solver}
\PYG{n}{of}\PYG{o}{.}\PYG{n}{stop}
\end{sphinxVerbatim}


\subsection{Full Tutorial Script}
\label{\detokenize{rst_tutorials/tutorial2_insitu:full-tutorial-script}}
To run the model, save your text file with the \sphinxtitleref{.of} extension. Rebuild the openfdem solution and drag your
\sphinxtitleref{.of} file into the \sphinxtitleref{OpenFDEM.exe}. It will automatically run and save the outputs.

\begin{sphinxVerbatim}[commandchars=\\\{\}]
\PYG{c+c1}{\PYGZsh{} Create a retangular block, group tag is rock, the range is xmin =0, xmax =1, ymin=0,ymax= 1}
\PYG{n}{of}\PYG{o}{.}\PYG{n}{geometry}\PYG{o}{.}\PYG{n}{square} \PYG{l+s+s1}{\PYGZsq{}}\PYG{l+s+s1}{rock}\PYG{l+s+s1}{\PYGZsq{}} \PYG{l+m+mi}{0} \PYG{l+m+mi}{1} \PYG{l+m+mi}{0} \PYG{l+m+mi}{1}
\PYG{c+c1}{\PYGZsh{} Create a hole in the block, cut is to fragment the block and seprate the rock block to hole block and }
\PYG{c+c1}{\PYGZsh{} new rock block (out of the hole block)}
\PYG{n}{of}\PYG{o}{.}\PYG{n}{geometry}\PYG{o}{.}\PYG{n}{cut}\PYG{o}{.}\PYG{n}{circle} \PYG{l+s+s1}{\PYGZsq{}}\PYG{l+s+s1}{hole}\PYG{l+s+s1}{\PYGZsq{}} \PYG{l+s+s1}{\PYGZsq{}}\PYG{l+s+s1}{rock}\PYG{l+s+s1}{\PYGZsq{}} \PYG{l+m+mf}{0.5} \PYG{l+m+mf}{0.5} \PYG{l+m+mf}{0.1} \PYG{l+m+mi}{70}
\PYG{c+c1}{\PYGZsh{} assign global mesh size, the default keyword is for global entities}
\PYG{n}{of}\PYG{o}{.}\PYG{n}{geometry}\PYG{o}{.}\PYG{n}{mesh}\PYG{o}{.}\PYG{n}{size} \PYG{l+s+s1}{\PYGZsq{}}\PYG{l+s+s1}{all}\PYG{l+s+s1}{\PYGZsq{}} \PYG{l+m+mf}{0.02}
\PYG{c+c1}{\PYGZsh{} starts to mesh using auto method, delaunay the default value}
\PYG{n}{of}\PYG{o}{.}\PYG{n}{geometry}\PYG{o}{.}\PYG{n}{mesh} \PYG{n}{auto}

\PYG{c+c1}{\PYGZsh{}create excavation element group using cricle tool, based on the centric point and radius}
\PYG{n}{of}\PYG{o}{.}\PYG{n}{group}\PYG{o}{.}\PYG{n}{element}\PYG{o}{.}\PYG{n}{circle} \PYG{l+s+s1}{\PYGZsq{}}\PYG{l+s+s1}{excavation}\PYG{l+s+s1}{\PYGZsq{}} \PYG{l+m+mf}{0.5} \PYG{l+m+mf}{0.5} \PYG{l+m+mf}{0.1}

\PYG{c+c1}{\PYGZsh{} assign  material parameters to solid elements based on the element groups}
\PYG{n}{of}\PYG{o}{.}\PYG{n}{mat}\PYG{o}{.}\PYG{n}{element} \PYG{l+s+s1}{\PYGZsq{}}\PYG{l+s+s1}{all}\PYG{l+s+s1}{\PYGZsq{}} \PYG{n}{ELASTIC} \PYG{n}{den} \PYG{l+m+mi}{2000} \PYG{n}{E} \PYG{l+m+mf}{30e9} \PYG{n}{v} \PYG{l+m+mf}{0.3} \PYG{n}{damp} \PYG{l+m+mf}{0.9}
\PYG{n}{of}\PYG{o}{.}\PYG{n}{mat}\PYG{o}{.}\PYG{n}{contact} \PYG{l+s+s1}{\PYGZsq{}}\PYG{l+s+s1}{all}\PYG{l+s+s1}{\PYGZsq{}} \PYG{n}{MC} \PYG{n}{fric} \PYG{l+m+mf}{0.3}

\PYG{c+c1}{\PYGZsh{} assign the in\PYGZhy{}situ stress in the domain}
\PYG{n}{of}\PYG{o}{.}\PYG{n}{boundary}\PYG{o}{.}\PYG{n}{element}\PYG{o}{.}\PYG{n}{stress} \PYG{o}{\PYGZhy{}}\PYG{l+m+mf}{35e6} \PYG{l+m+mf}{0.0} \PYG{o}{\PYGZhy{}}\PYG{l+m+mf}{35e6}

\PYG{c+c1}{\PYGZsh{} create nodal groups on edges}
\PYG{n}{of}\PYG{o}{.}\PYG{n}{group}\PYG{o}{.}\PYG{n}{nodal}\PYG{o}{.}\PYG{n}{plane} \PYG{l+s+s1}{\PYGZsq{}}\PYG{l+s+s1}{bottom\PYGZus{}edge}\PYG{l+s+s1}{\PYGZsq{}} \PYG{l+m+mf}{0.0} \PYG{l+m+mf}{0.0} \PYG{l+m+mf}{1.0} \PYG{l+m+mf}{0.0}
\PYG{n}{of}\PYG{o}{.}\PYG{n}{group}\PYG{o}{.}\PYG{n}{nodal}\PYG{o}{.}\PYG{n}{plane} \PYG{l+s+s1}{\PYGZsq{}}\PYG{l+s+s1}{up\PYGZus{}edge}\PYG{l+s+s1}{\PYGZsq{}} \PYG{l+m+mf}{0.0} \PYG{l+m+mf}{1.0} \PYG{l+m+mf}{1.0} \PYG{l+m+mf}{1.0}
\PYG{n}{of}\PYG{o}{.}\PYG{n}{group}\PYG{o}{.}\PYG{n}{nodal}\PYG{o}{.}\PYG{n}{plane} \PYG{l+s+s1}{\PYGZsq{}}\PYG{l+s+s1}{left\PYGZus{}edge}\PYG{l+s+s1}{\PYGZsq{}} \PYG{l+m+mf}{0.0} \PYG{l+m+mf}{0.0} \PYG{l+m+mf}{0.0} \PYG{l+m+mf}{1.0}
\PYG{n}{of}\PYG{o}{.}\PYG{n}{group}\PYG{o}{.}\PYG{n}{nodal}\PYG{o}{.}\PYG{n}{plane} \PYG{l+s+s1}{\PYGZsq{}}\PYG{l+s+s1}{right\PYGZus{}edge}\PYG{l+s+s1}{\PYGZsq{}} \PYG{l+m+mf}{1.0} \PYG{l+m+mf}{0.0} \PYG{l+m+mf}{1.0} \PYG{l+m+mf}{1.0}

\PYG{c+c1}{\PYGZsh{} set interval to write paraview field results}
\PYG{n}{of}\PYG{o}{.}\PYG{n}{history}\PYG{o}{.}\PYG{n}{pv}\PYG{o}{.}\PYG{n}{interval} \PYG{l+m+mi}{500}
\PYG{n}{of}\PYG{o}{.}\PYG{n}{history}\PYG{o}{.}\PYG{n}{pv}\PYG{o}{.}\PYG{n}{field} \PYG{n+nb}{all}
\PYG{n}{of}\PYG{o}{.}\PYG{n}{history}\PYG{o}{.}\PYG{n}{pv}\PYG{o}{.}\PYG{n}{fracture} \PYG{n+nb}{all}
\PYG{n}{of}\PYG{o}{.}\PYG{n}{history}\PYG{o}{.}\PYG{n}{pv}\PYG{o}{.}\PYG{n}{cohesive} \PYG{n+nb}{all}

\PYG{c+c1}{\PYGZsh{} run steps to equlibrum the in\PYGZhy{}situ stress, it should be fast}
\PYG{n}{of}\PYG{o}{.}\PYG{n}{step} \PYG{l+m+mi}{1000}

\PYG{c+c1}{\PYGZsh{} insert cohesive elements, the cohesive elements should be inserted fater in\PYGZhy{}situ stress}
\PYG{n}{of}\PYG{o}{.}\PYG{n}{mesh}\PYG{o}{.}\PYG{n}{insert} \PYG{l+s+s1}{\PYGZsq{}}\PYG{l+s+s1}{all}\PYG{l+s+s1}{\PYGZsq{}}
\PYG{c+c1}{\PYGZsh{} assign material parameters for cohesive elements}
\PYG{n}{of}\PYG{o}{.}\PYG{n}{mat}\PYG{o}{.}\PYG{n}{cohesive} \PYG{l+s+s1}{\PYGZsq{}}\PYG{l+s+s1}{all}\PYG{l+s+s1}{\PYGZsq{}} \PYG{n}{EM} \PYG{n}{ten} \PYG{l+m+mf}{10e6} \PYG{n}{coh} \PYG{l+m+mf}{20e6} \PYG{n}{fric} \PYG{l+m+mf}{0.3} \PYG{n}{GI} \PYG{l+m+mi}{20} \PYG{n}{GII} \PYG{l+m+mi}{40} 

\PYG{c+c1}{\PYGZsh{} excavte the tunnel}
\PYG{n}{of}\PYG{o}{.}\PYG{n}{boundary}\PYG{o}{.}\PYG{n}{excavation} \PYG{l+s+s1}{\PYGZsq{}}\PYG{l+s+s1}{excavation}\PYG{l+s+s1}{\PYGZsq{}}
\PYG{c+c1}{\PYGZsh{} fix the outer boundaries}
\PYG{n}{of}\PYG{o}{.}\PYG{n}{boundary}\PYG{o}{.}\PYG{n}{nodal}\PYG{o}{.}\PYG{n}{velocity} \PYG{l+s+s1}{\PYGZsq{}}\PYG{l+s+s1}{bottom}\PYG{l+s+s1}{\PYGZsq{}} \PYG{n}{Y} \PYG{l+m+mi}{0} 
\PYG{n}{of}\PYG{o}{.}\PYG{n}{boundary}\PYG{o}{.}\PYG{n}{nodal}\PYG{o}{.}\PYG{n}{velocity} \PYG{l+s+s1}{\PYGZsq{}}\PYG{l+s+s1}{up}\PYG{l+s+s1}{\PYGZsq{}} \PYG{n}{Y} \PYG{l+m+mi}{0} 
\PYG{n}{of}\PYG{o}{.}\PYG{n}{boundary}\PYG{o}{.}\PYG{n}{nodal}\PYG{o}{.}\PYG{n}{velocity} \PYG{l+s+s1}{\PYGZsq{}}\PYG{l+s+s1}{left}\PYG{l+s+s1}{\PYGZsq{}} \PYG{n}{X} \PYG{l+m+mi}{0} 
\PYG{n}{of}\PYG{o}{.}\PYG{n}{boundary}\PYG{o}{.}\PYG{n}{nodal}\PYG{o}{.}\PYG{n}{velocity} \PYG{l+s+s1}{\PYGZsq{}}\PYG{l+s+s1}{right}\PYG{l+s+s1}{\PYGZsq{}} \PYG{n}{X} \PYG{l+m+mi}{0} 

\PYG{n}{of}\PYG{o}{.}\PYG{n}{step} \PYG{l+m+mi}{50000}
\PYG{c+c1}{\PYGZsh{} terminate the run and step out solver}
\PYG{n}{of}\PYG{o}{.}\PYG{n}{stop}
\end{sphinxVerbatim}


\chapter{Tutorial 3: Pressure Method}
\label{\detokenize{rst_tutorials/tutorial3_pressure:tutorial-3-pressure-method}}\label{\detokenize{rst_tutorials/tutorial3_pressure::doc}}
\sphinxstylestrong{Runtime}: \textasciitilde{}2 hours on i9 8\sphinxhyphen{}core Windows 10 Machine


\section{Full Tutorial Script}
\label{\detokenize{rst_tutorials/tutorial3_pressure:full-tutorial-script}}
To run the model, save your text file with the \sphinxtitleref{.of} extension. Rebuild the openfdem solution and drag your
\sphinxtitleref{.of} file into the \sphinxtitleref{OpenFDEM.exe}. It will automatically run and save the outputs.

\begin{sphinxVerbatim}[commandchars=\\\{\}]
\PYG{c+c1}{\PYGZsh{} clear old memories, be optional}
\PYG{n}{of}\PYG{o}{.}\PYG{n}{new}

\PYG{c+c1}{\PYGZsh{} Create a retangular block, group tag is rock, the range is xmin =0, xmax =1, ymin=0,ymax= 1}
\PYG{n}{of}\PYG{o}{.}\PYG{n}{geometry}\PYG{o}{.}\PYG{n}{square} \PYG{l+s+s1}{\PYGZsq{}}\PYG{l+s+s1}{rock}\PYG{l+s+s1}{\PYGZsq{}} \PYG{l+m+mi}{0} \PYG{l+m+mi}{1} \PYG{l+m+mi}{0} \PYG{l+m+mi}{1}
\PYG{c+c1}{\PYGZsh{} Create a hole in the block, cut is to fragment the block and seprate the rock block to hole block and }
\PYG{c+c1}{\PYGZsh{} new rock block (out of the hole block)}
\PYG{n}{of}\PYG{o}{.}\PYG{n}{geometry}\PYG{o}{.}\PYG{n}{cut}\PYG{o}{.}\PYG{n}{circle} \PYG{l+s+s1}{\PYGZsq{}}\PYG{l+s+s1}{hole}\PYG{l+s+s1}{\PYGZsq{}} \PYG{l+s+s1}{\PYGZsq{}}\PYG{l+s+s1}{rock}\PYG{l+s+s1}{\PYGZsq{}} \PYG{l+m+mf}{0.5} \PYG{l+m+mf}{0.5} \PYG{l+m+mf}{0.1} \PYG{l+m+mi}{70}
\PYG{c+c1}{\PYGZsh{} assign global mesh size, the default keyword is for global entities}
\PYG{n}{of}\PYG{o}{.}\PYG{n}{geometry}\PYG{o}{.}\PYG{n}{mesh}\PYG{o}{.}\PYG{n}{size} \PYG{l+s+s1}{\PYGZsq{}}\PYG{l+s+s1}{all}\PYG{l+s+s1}{\PYGZsq{}} \PYG{l+m+mf}{0.02}
\PYG{c+c1}{\PYGZsh{} starts to mesh, delaunay is optional, it is the default value}
\PYG{n}{of}\PYG{o}{.}\PYG{n}{geometry}\PYG{o}{.}\PYG{n}{mesh} \PYG{n}{delaunay}

\PYG{c+c1}{\PYGZsh{}insert cohesive elements, globally}
\PYG{n}{of}\PYG{o}{.}\PYG{n}{mesh}\PYG{o}{.}\PYG{n}{insert} \PYG{l+s+s1}{\PYGZsq{}}\PYG{l+s+s1}{all}\PYG{l+s+s1}{\PYGZsq{}}

\PYG{c+c1}{\PYGZsh{} assign  material parameters to solid elements based on the element groups}
\PYG{n}{of}\PYG{o}{.}\PYG{n}{mat}\PYG{o}{.}\PYG{n}{element} \PYG{l+s+s1}{\PYGZsq{}}\PYG{l+s+s1}{all}\PYG{l+s+s1}{\PYGZsq{}} \PYG{n}{elastic} \PYG{n}{den} \PYG{l+m+mi}{2000} \PYG{n}{E} \PYG{l+m+mf}{30e9} \PYG{n}{v} \PYG{l+m+mf}{0.3} \PYG{n}{damp} \PYG{l+m+mf}{0.9}
\PYG{n}{of}\PYG{o}{.}\PYG{n}{mat}\PYG{o}{.}\PYG{n}{cohesive} \PYG{l+s+s1}{\PYGZsq{}}\PYG{l+s+s1}{all}\PYG{l+s+s1}{\PYGZsq{}} \PYG{n}{EM} \PYG{n}{ten} \PYG{l+m+mf}{10e6} \PYG{n}{coh} \PYG{l+m+mf}{20e6} \PYG{n}{fric} \PYG{l+m+mf}{0.3} \PYG{n}{GI} \PYG{l+m+mi}{20} \PYG{n}{GII} \PYG{l+m+mi}{40} 
\PYG{n}{of}\PYG{o}{.}\PYG{n}{mat}\PYG{o}{.}\PYG{n}{contact} \PYG{l+s+s1}{\PYGZsq{}}\PYG{l+s+s1}{all}\PYG{l+s+s1}{\PYGZsq{}} \PYG{n}{MC} \PYG{n}{fric} \PYG{l+m+mf}{0.3}

\PYG{c+c1}{\PYGZsh{}create excavation element group using cricle tool, based on the centric point and radius}
\PYG{n}{of}\PYG{o}{.}\PYG{n}{group}\PYG{o}{.}\PYG{n}{element}\PYG{o}{.}\PYG{n}{circle} \PYG{l+s+s1}{\PYGZsq{}}\PYG{l+s+s1}{excavation}\PYG{l+s+s1}{\PYGZsq{}} \PYG{l+m+mf}{0.5} \PYG{l+m+mf}{0.5} \PYG{l+m+mf}{0.1}
\PYG{c+c1}{\PYGZsh{}create bottom\PYGZus{}edge nodal group using plane tool, based on the start point coord and end point coord}
\PYG{n}{of}\PYG{o}{.}\PYG{n}{group}\PYG{o}{.}\PYG{n}{nodal}\PYG{o}{.}\PYG{n}{plane} \PYG{l+s+s1}{\PYGZsq{}}\PYG{l+s+s1}{bottom\PYGZus{}edge}\PYG{l+s+s1}{\PYGZsq{}} \PYG{l+m+mf}{0.0} \PYG{l+m+mf}{0.0} \PYG{l+m+mf}{1.0} \PYG{l+m+mf}{0.0}
\PYG{c+c1}{\PYGZsh{}create up\PYGZus{}edge nodal group using plane tool, based on the start point coord and end point coord}
\PYG{n}{of}\PYG{o}{.}\PYG{n}{group}\PYG{o}{.}\PYG{n}{nodal}\PYG{o}{.}\PYG{n}{plane} \PYG{l+s+s1}{\PYGZsq{}}\PYG{l+s+s1}{up\PYGZus{}edge}\PYG{l+s+s1}{\PYGZsq{}} \PYG{l+m+mf}{0.0} \PYG{l+m+mf}{1.0} \PYG{l+m+mf}{1.0} \PYG{l+m+mf}{1.0}
\PYG{c+c1}{\PYGZsh{}create left\PYGZus{}edge nodal group using plane tool, based on the start point coord and end point coord}
\PYG{n}{of}\PYG{o}{.}\PYG{n}{group}\PYG{o}{.}\PYG{n}{nodal}\PYG{o}{.}\PYG{n}{plane} \PYG{l+s+s1}{\PYGZsq{}}\PYG{l+s+s1}{left\PYGZus{}edge}\PYG{l+s+s1}{\PYGZsq{}} \PYG{l+m+mf}{0.0} \PYG{l+m+mf}{0.0} \PYG{l+m+mf}{0.0} \PYG{l+m+mf}{1.0}
\PYG{c+c1}{\PYGZsh{}create right\PYGZus{}edge nodal group using plane tool, based on the start point coord and end point coord}
\PYG{n}{of}\PYG{o}{.}\PYG{n}{group}\PYG{o}{.}\PYG{n}{nodal}\PYG{o}{.}\PYG{n}{plane} \PYG{l+s+s1}{\PYGZsq{}}\PYG{l+s+s1}{right\PYGZus{}edge}\PYG{l+s+s1}{\PYGZsq{}} \PYG{l+m+mf}{1.0} \PYG{l+m+mf}{0.0} \PYG{l+m+mf}{1.0} \PYG{l+m+mf}{1.0}

\PYG{c+c1}{\PYGZsh{} set interval to write paraview field results}
\PYG{n}{of}\PYG{o}{.}\PYG{n}{history}\PYG{o}{.}\PYG{n}{pv}\PYG{o}{.}\PYG{n}{interval} \PYG{l+m+mi}{500}
\PYG{n}{of}\PYG{o}{.}\PYG{n}{history}\PYG{o}{.}\PYG{n}{pv}\PYG{o}{.}\PYG{n}{field} \PYG{n+nb}{all}
\PYG{n}{of}\PYG{o}{.}\PYG{n}{history}\PYG{o}{.}\PYG{n}{pv}\PYG{o}{.}\PYG{n}{fracture} \PYG{n+nb}{all}
\PYG{n}{of}\PYG{o}{.}\PYG{n}{history}\PYG{o}{.}\PYG{n}{pv}\PYG{o}{.}\PYG{n}{cohesive} \PYG{n+nb}{all}

\PYG{c+c1}{\PYGZsh{} assign pressure boundary on edges, comprssion is postive}
\PYG{n}{of}\PYG{o}{.}\PYG{n}{boundary}\PYG{o}{.}\PYG{n}{pressure}\PYG{o}{.}\PYG{n}{normal} \PYG{l+s+s1}{\PYGZsq{}}\PYG{l+s+s1}{bottom\PYGZus{}edge}\PYG{l+s+s1}{\PYGZsq{}} \PYG{l+m+mf}{15e6}
\PYG{n}{of}\PYG{o}{.}\PYG{n}{boundary}\PYG{o}{.}\PYG{n}{pressure}\PYG{o}{.}\PYG{n}{normal} \PYG{l+s+s1}{\PYGZsq{}}\PYG{l+s+s1}{up\PYGZus{}edge}\PYG{l+s+s1}{\PYGZsq{}} \PYG{l+m+mf}{15e6}
\PYG{n}{of}\PYG{o}{.}\PYG{n}{boundary}\PYG{o}{.}\PYG{n}{pressure}\PYG{o}{.}\PYG{n}{normal} \PYG{l+s+s1}{\PYGZsq{}}\PYG{l+s+s1}{left\PYGZus{}edge}\PYG{l+s+s1}{\PYGZsq{}} \PYG{l+m+mf}{5e6}
\PYG{n}{of}\PYG{o}{.}\PYG{n}{boundary}\PYG{o}{.}\PYG{n}{pressure}\PYG{o}{.}\PYG{n}{normal} \PYG{l+s+s1}{\PYGZsq{}}\PYG{l+s+s1}{right\PYGZus{}edge}\PYG{l+s+s1}{\PYGZsq{}} \PYG{l+m+mf}{5e6}

\PYG{c+c1}{\PYGZsh{} total run steps to equlibrium the insitu stress}
\PYG{n}{of}\PYG{o}{.}\PYG{n}{step} \PYG{l+m+mi}{100000}

\PYG{c+c1}{\PYGZsh{} excavted the hole}
\PYG{n}{of}\PYG{o}{.}\PYG{n}{boundary}\PYG{o}{.}\PYG{n}{excavation} \PYG{l+s+s1}{\PYGZsq{}}\PYG{l+s+s1}{excavation}\PYG{l+s+s1}{\PYGZsq{}}
\PYG{c+c1}{\PYGZsh{} run steps to compute the tunnel deformation}
\PYG{n}{of}\PYG{o}{.}\PYG{n}{step} \PYG{l+m+mi}{500000}
\PYG{c+c1}{\PYGZsh{} terminate the run and step out solver}
\PYG{n}{of}\PYG{o}{.}\PYG{n}{stop}
\end{sphinxVerbatim}

OpenFDEM input files use a Python\sphinxhyphen{}like format and the dot is used to
clarify the hierarchy of different classes.

The first layer is OpenFDEM, declared as \sphinxcode{\sphinxupquote{.of}} or \sphinxcode{\sphinxupquote{.OpenFDEM}}, but the
commands in other layers cannot be shortened (for example, of.nodal or
OpenFDEM.nodal).

\begin{sphinxShadowBox}
\sphinxstyletopictitle{Table of Contents}
\begin{itemize}
\item {} 
\phantomsection\label{\detokenize{rst_tutorials/command_line_guide:id6}}{\hyperref[\detokenize{rst_tutorials/command_line_guide:general-commands}]{\sphinxcrossref{general commands}}}
\begin{itemize}
\item {} 
\phantomsection\label{\detokenize{rst_tutorials/command_line_guide:id7}}{\hyperref[\detokenize{rst_tutorials/command_line_guide:of-new}]{\sphinxcrossref{of.new}}}

\item {} 
\phantomsection\label{\detokenize{rst_tutorials/command_line_guide:id8}}{\hyperref[\detokenize{rst_tutorials/command_line_guide:of-restore}]{\sphinxcrossref{of.restore}}}

\item {} 
\phantomsection\label{\detokenize{rst_tutorials/command_line_guide:id9}}{\hyperref[\detokenize{rst_tutorials/command_line_guide:of-result}]{\sphinxcrossref{of.result}}}

\item {} 
\phantomsection\label{\detokenize{rst_tutorials/command_line_guide:id10}}{\hyperref[\detokenize{rst_tutorials/command_line_guide:of-save}]{\sphinxcrossref{of.save}}}

\item {} 
\phantomsection\label{\detokenize{rst_tutorials/command_line_guide:id11}}{\hyperref[\detokenize{rst_tutorials/command_line_guide:of-import}]{\sphinxcrossref{of.import}}}

\item {} 
\phantomsection\label{\detokenize{rst_tutorials/command_line_guide:id12}}{\hyperref[\detokenize{rst_tutorials/command_line_guide:of-debug-on}]{\sphinxcrossref{of.debug on}}}

\end{itemize}

\item {} 
\phantomsection\label{\detokenize{rst_tutorials/command_line_guide:id13}}{\hyperref[\detokenize{rst_tutorials/command_line_guide:config}]{\sphinxcrossref{config}}}
\begin{itemize}
\item {} 
\phantomsection\label{\detokenize{rst_tutorials/command_line_guide:id14}}{\hyperref[\detokenize{rst_tutorials/command_line_guide:of-config-type-planestress-planestrain}]{\sphinxcrossref{of.config.type planestress/planestrain}}}

\item {} 
\phantomsection\label{\detokenize{rst_tutorials/command_line_guide:id15}}{\hyperref[\detokenize{rst_tutorials/command_line_guide:of-config-module-hydro-blast-thermal}]{\sphinxcrossref{of.config.module hydro/blast/thermal}}}

\end{itemize}

\item {} 
\phantomsection\label{\detokenize{rst_tutorials/command_line_guide:id16}}{\hyperref[\detokenize{rst_tutorials/command_line_guide:geometry}]{\sphinxcrossref{geometry}}}
\begin{itemize}
\item {} 
\phantomsection\label{\detokenize{rst_tutorials/command_line_guide:id17}}{\hyperref[\detokenize{rst_tutorials/command_line_guide:of-geometry-square-name-x0-y0-x1-y1-x2-y2-x3-y3}]{\sphinxcrossref{of.geometry.square name x0 y0 x1 y1 x2 y2 x3 y3}}}

\item {} 
\phantomsection\label{\detokenize{rst_tutorials/command_line_guide:id18}}{\hyperref[\detokenize{rst_tutorials/command_line_guide:of-geometry-polygon-name-n-x0-y0-x1-y1-x2-y2-x3-y3}]{\sphinxcrossref{of.geometry.polygon name n x0 y0 x1 y1 x2 y2 x3 y3}}}

\item {} 
\phantomsection\label{\detokenize{rst_tutorials/command_line_guide:id19}}{\hyperref[\detokenize{rst_tutorials/command_line_guide:of-geometry-table-name-table-tab}]{\sphinxcrossref{of.geometry.table name ‘table.tab’}}}

\item {} 
\phantomsection\label{\detokenize{rst_tutorials/command_line_guide:id20}}{\hyperref[\detokenize{rst_tutorials/command_line_guide:of-geometry-circle-name-x0-y0-r-n}]{\sphinxcrossref{of.geometry.circle name x0 y0 r n}}}

\item {} 
\phantomsection\label{\detokenize{rst_tutorials/command_line_guide:id21}}{\hyperref[\detokenize{rst_tutorials/command_line_guide:of-geometry-ellipse-name-x0-y0-rmax-rmin-theta-n}]{\sphinxcrossref{of.geometry.ellipse name x0 y0 rmax rmin theta n}}}

\item {} 
\phantomsection\label{\detokenize{rst_tutorials/command_line_guide:id22}}{\hyperref[\detokenize{rst_tutorials/command_line_guide:of-geometry-cut-square-name-name2-x0-y0-x1-y1-x2-y2-x3-y3}]{\sphinxcrossref{of.geometry.cut.square name name2 x0 y0 x1 y1 x2 y2 x3 y3}}}

\item {} 
\phantomsection\label{\detokenize{rst_tutorials/command_line_guide:id23}}{\hyperref[\detokenize{rst_tutorials/command_line_guide:of-geometry-cut-polygon-name-name2-n-x0-y0-x1-y1-x2-y2-x3-y3}]{\sphinxcrossref{of.geometry.cut.polygon name name2 n x0 y0 x1 y1 x2 y2 x3 y3}}}

\item {} 
\phantomsection\label{\detokenize{rst_tutorials/command_line_guide:id24}}{\hyperref[\detokenize{rst_tutorials/command_line_guide:of-geometry-cut-table-name-name2-table-tab}]{\sphinxcrossref{of.geometry.cut.table name name2 ‘table.tab’}}}

\item {} 
\phantomsection\label{\detokenize{rst_tutorials/command_line_guide:id25}}{\hyperref[\detokenize{rst_tutorials/command_line_guide:of-geometry-cut-circle-name-name2-x0-y0-r-n}]{\sphinxcrossref{of.geometry.cut.circle name name2 x0 y0 r n}}}

\item {} 
\phantomsection\label{\detokenize{rst_tutorials/command_line_guide:id26}}{\hyperref[\detokenize{rst_tutorials/command_line_guide:of-geometry-cut-ellipse-name-name2-x0-y0-rmax-rmin-theta-n}]{\sphinxcrossref{of.geometry.cut.ellipse name name2 x0 y0 rmax rmin theta n}}}

\item {} 
\phantomsection\label{\detokenize{rst_tutorials/command_line_guide:id27}}{\hyperref[\detokenize{rst_tutorials/command_line_guide:of-geometry-cut-jset-name-name2-dip-length-space-gap}]{\sphinxcrossref{of.geometry.cut.jset name name2 dip length space gap}}}

\item {} 
\phantomsection\label{\detokenize{rst_tutorials/command_line_guide:id28}}{\hyperref[\detokenize{rst_tutorials/command_line_guide:of-geometry-cut-dfn-name-name2-method-dip-length}]{\sphinxcrossref{of.geometry.cut.dfn name name2 method dip length}}}

\item {} 
\phantomsection\label{\detokenize{rst_tutorials/command_line_guide:id29}}{\hyperref[\detokenize{rst_tutorials/command_line_guide:of-geometry-import-rdfn-pixel-ratio-name-name2-name}]{\sphinxcrossref{of.geometry.import.rdfn pixel\_ratio name name2 name}}}

\item {} 
\phantomsection\label{\detokenize{rst_tutorials/command_line_guide:id30}}{\hyperref[\detokenize{rst_tutorials/command_line_guide:of-geometry-remove-square-name-name2-x0-y0-x1-y1-x2-y2-x3-y3}]{\sphinxcrossref{of.geometry.remove.square name name2 x0 y0 x1 y1 x2 y2 x3 y3}}}

\item {} 
\phantomsection\label{\detokenize{rst_tutorials/command_line_guide:id31}}{\hyperref[\detokenize{rst_tutorials/command_line_guide:of-geometry-remove-polygon-name-name2-n-x0-y0-x1-y1-x2-y2-x3-y3}]{\sphinxcrossref{of.geometry.remove.polygon name name2 n x0 y0 x1 y1 x2 y2 x3 y3}}}

\item {} 
\phantomsection\label{\detokenize{rst_tutorials/command_line_guide:id32}}{\hyperref[\detokenize{rst_tutorials/command_line_guide:of-geometry-remove-table-name-name2-table-tab}]{\sphinxcrossref{of.geometry.remove.table name name2 ‘table.tab’}}}

\item {} 
\phantomsection\label{\detokenize{rst_tutorials/command_line_guide:id33}}{\hyperref[\detokenize{rst_tutorials/command_line_guide:of-geometry-remove-circle-name-name2-x1-y1-r-n}]{\sphinxcrossref{of.geometry.remove.circle name name2 x1 y1 r n}}}

\item {} 
\phantomsection\label{\detokenize{rst_tutorials/command_line_guide:id34}}{\hyperref[\detokenize{rst_tutorials/command_line_guide:of-geometry-remove-ellipse-name-name2-x0-y0-rmax-rmin-theta-n}]{\sphinxcrossref{of.geometry.remove.ellipse name name2 x0 y0 rmax rmin theta n}}}

\item {} 
\phantomsection\label{\detokenize{rst_tutorials/command_line_guide:id35}}{\hyperref[\detokenize{rst_tutorials/command_line_guide:of-geometry-mesh-size-name-default}]{\sphinxcrossref{of.geometry.mesh.size name default}}}

\item {} 
\phantomsection\label{\detokenize{rst_tutorials/command_line_guide:id36}}{\hyperref[\detokenize{rst_tutorials/command_line_guide:of-geometry-mesh-keyword}]{\sphinxcrossref{of.geometry.mesh keyword}}}

\end{itemize}

\item {} 
\phantomsection\label{\detokenize{rst_tutorials/command_line_guide:id37}}{\hyperref[\detokenize{rst_tutorials/command_line_guide:mesh-insert}]{\sphinxcrossref{mesh\_insert}}}
\begin{itemize}
\item {} 
\phantomsection\label{\detokenize{rst_tutorials/command_line_guide:id38}}{\hyperref[\detokenize{rst_tutorials/command_line_guide:of-mesh-insert-elementgroup}]{\sphinxcrossref{of.mesh.insert elementgroup}}}

\end{itemize}

\item {} 
\phantomsection\label{\detokenize{rst_tutorials/command_line_guide:id39}}{\hyperref[\detokenize{rst_tutorials/command_line_guide:nodal}]{\sphinxcrossref{nodal}}}
\begin{itemize}
\item {} 
\phantomsection\label{\detokenize{rst_tutorials/command_line_guide:id40}}{\hyperref[\detokenize{rst_tutorials/command_line_guide:of-nodal-coord-number-nodes}]{\sphinxcrossref{of.nodal.coord number\_nodes}}}

\end{itemize}

\item {} 
\phantomsection\label{\detokenize{rst_tutorials/command_line_guide:id41}}{\hyperref[\detokenize{rst_tutorials/command_line_guide:element}]{\sphinxcrossref{element}}}
\begin{itemize}
\item {} 
\phantomsection\label{\detokenize{rst_tutorials/command_line_guide:id42}}{\hyperref[\detokenize{rst_tutorials/command_line_guide:of-element-connectivity-number-nodes-number-elements}]{\sphinxcrossref{of.element.connectivity number\_nodes number\_elements}}}

\end{itemize}

\item {} 
\phantomsection\label{\detokenize{rst_tutorials/command_line_guide:id43}}{\hyperref[\detokenize{rst_tutorials/command_line_guide:contact}]{\sphinxcrossref{contact}}}
\begin{itemize}
\item {} 
\phantomsection\label{\detokenize{rst_tutorials/command_line_guide:id44}}{\hyperref[\detokenize{rst_tutorials/command_line_guide:of-contact-detection-nbs}]{\sphinxcrossref{of.contact.detection nbs}}}

\item {} 
\phantomsection\label{\detokenize{rst_tutorials/command_line_guide:id45}}{\hyperref[\detokenize{rst_tutorials/command_line_guide:of-contact-force-lig}]{\sphinxcrossref{of.contact.force lig}}}

\end{itemize}

\item {} 
\phantomsection\label{\detokenize{rst_tutorials/command_line_guide:id46}}{\hyperref[\detokenize{rst_tutorials/command_line_guide:cohelement}]{\sphinxcrossref{cohelement}}}
\begin{itemize}
\item {} 
\phantomsection\label{\detokenize{rst_tutorials/command_line_guide:id47}}{\hyperref[\detokenize{rst_tutorials/command_line_guide:of-cohelement-delete-cohelementgroup}]{\sphinxcrossref{of.cohelement.delete cohelementgroup}}}

\end{itemize}

\item {} 
\phantomsection\label{\detokenize{rst_tutorials/command_line_guide:id48}}{\hyperref[\detokenize{rst_tutorials/command_line_guide:group}]{\sphinxcrossref{group}}}
\begin{itemize}
\item {} 
\phantomsection\label{\detokenize{rst_tutorials/command_line_guide:id49}}{\hyperref[\detokenize{rst_tutorials/command_line_guide:nodal-groups}]{\sphinxcrossref{nodal groups}}}

\item {} 
\phantomsection\label{\detokenize{rst_tutorials/command_line_guide:id50}}{\hyperref[\detokenize{rst_tutorials/command_line_guide:bool}]{\sphinxcrossref{bool}}}

\item {} 
\phantomsection\label{\detokenize{rst_tutorials/command_line_guide:id51}}{\hyperref[\detokenize{rst_tutorials/command_line_guide:element-1}]{\sphinxcrossref{element}}}

\item {} 
\phantomsection\label{\detokenize{rst_tutorials/command_line_guide:id52}}{\hyperref[\detokenize{rst_tutorials/command_line_guide:cohesive-element}]{\sphinxcrossref{cohesive element}}}

\end{itemize}

\item {} 
\phantomsection\label{\detokenize{rst_tutorials/command_line_guide:id53}}{\hyperref[\detokenize{rst_tutorials/command_line_guide:boundary}]{\sphinxcrossref{boundary}}}
\begin{itemize}
\item {} 
\phantomsection\label{\detokenize{rst_tutorials/command_line_guide:id54}}{\hyperref[\detokenize{rst_tutorials/command_line_guide:nodal-1}]{\sphinxcrossref{nodal}}}

\item {} 
\phantomsection\label{\detokenize{rst_tutorials/command_line_guide:id55}}{\hyperref[\detokenize{rst_tutorials/command_line_guide:nodal-local-edge}]{\sphinxcrossref{nodal local edge}}}

\item {} 
\phantomsection\label{\detokenize{rst_tutorials/command_line_guide:id56}}{\hyperref[\detokenize{rst_tutorials/command_line_guide:nodal-table}]{\sphinxcrossref{nodal table}}}

\item {} 
\phantomsection\label{\detokenize{rst_tutorials/command_line_guide:id57}}{\hyperref[\detokenize{rst_tutorials/command_line_guide:hydro}]{\sphinxcrossref{hydro}}}

\item {} 
\phantomsection\label{\detokenize{rst_tutorials/command_line_guide:id58}}{\hyperref[\detokenize{rst_tutorials/command_line_guide:blast}]{\sphinxcrossref{blast}}}

\item {} 
\phantomsection\label{\detokenize{rst_tutorials/command_line_guide:id59}}{\hyperref[\detokenize{rst_tutorials/command_line_guide:id4}]{\sphinxcrossref{element}}}

\item {} 
\phantomsection\label{\detokenize{rst_tutorials/command_line_guide:id60}}{\hyperref[\detokenize{rst_tutorials/command_line_guide:thermal}]{\sphinxcrossref{thermal}}}

\end{itemize}

\item {} 
\phantomsection\label{\detokenize{rst_tutorials/command_line_guide:id61}}{\hyperref[\detokenize{rst_tutorials/command_line_guide:material}]{\sphinxcrossref{material}}}
\begin{itemize}
\item {} 
\phantomsection\label{\detokenize{rst_tutorials/command_line_guide:id62}}{\hyperref[\detokenize{rst_tutorials/command_line_guide:of-mat-element-elementgroupname-all-modelname-p1-p2-p3}]{\sphinxcrossref{of.mat.element elementgroupname(all) modelname p1 p2 p3 …}}}

\item {} 
\phantomsection\label{\detokenize{rst_tutorials/command_line_guide:id63}}{\hyperref[\detokenize{rst_tutorials/command_line_guide:of-mat-particle-elementgroupname-all-modelname-p1-p2-p3}]{\sphinxcrossref{of.mat.particle elementgroupname(all) modelname p1 p2 p3 …}}}

\item {} 
\phantomsection\label{\detokenize{rst_tutorials/command_line_guide:id64}}{\hyperref[\detokenize{rst_tutorials/command_line_guide:of-mat-cohesive-cohelementgroupname-all-modelname-p1-p2-p3}]{\sphinxcrossref{of.mat.cohesive cohelementgroupname(all) modelname p1 p2 p3 …}}}

\item {} 
\phantomsection\label{\detokenize{rst_tutorials/command_line_guide:id65}}{\hyperref[\detokenize{rst_tutorials/command_line_guide:of-mat-cohesive-rock-em-het-power-0-5-dip-35-ten-30e6-3e6-coh-30e6-3e6-fric-0-3-0-gi-100-20-gii-300-30-power-dip-mean-pn-dev-pn-mean-pt-dev-pt-mean-ten-dev-ten-mean-coh-dev-coh-mean-fri-dev-fri-mean-gi-dev-gi-mean-gii-dev-gii}]{\sphinxcrossref{of.mat. cohesive ‘rock’ em\_het power 0.5 dip 35 ten 30e6 3e6 coh 30e6 3e6 fric 0.3 0 gi 100 20 gii 300 30 (power, dip, mean pn, dev pn, mean pt, dev pt, mean ten, dev ten, mean coh, dev coh, mean fri, dev fri, mean gi, dev gi, mean gii, dev gii)}}}

\item {} 
\phantomsection\label{\detokenize{rst_tutorials/command_line_guide:id66}}{\hyperref[\detokenize{rst_tutorials/command_line_guide:of-mat-contact-elementgroupname1-elementgroupname2-all-modelname-p1-p2-p3}]{\sphinxcrossref{of.mat.contact elementgroupname1 elementgroupname2(all) modelname p1 p2 p3 …}}}

\end{itemize}

\item {} 
\phantomsection\label{\detokenize{rst_tutorials/command_line_guide:id67}}{\hyperref[\detokenize{rst_tutorials/command_line_guide:gbm}]{\sphinxcrossref{gbm}}}
\begin{itemize}
\item {} 
\phantomsection\label{\detokenize{rst_tutorials/command_line_guide:id68}}{\hyperref[\detokenize{rst_tutorials/command_line_guide:of-gbm-numberofminerals-m1-name-ratio-m2-name-ratio}]{\sphinxcrossref{of.gbm numberofminerals m1\_name ratio m2\_name ratio …}}}

\end{itemize}

\item {} 
\phantomsection\label{\detokenize{rst_tutorials/command_line_guide:id69}}{\hyperref[\detokenize{rst_tutorials/command_line_guide:history}]{\sphinxcrossref{history}}}
\begin{itemize}
\item {} 
\phantomsection\label{\detokenize{rst_tutorials/command_line_guide:id70}}{\hyperref[\detokenize{rst_tutorials/command_line_guide:of-history-nodal-force-id-x1-y1}]{\sphinxcrossref{of.history.nodal.force id x1 y1}}}

\item {} 
\phantomsection\label{\detokenize{rst_tutorials/command_line_guide:id71}}{\hyperref[\detokenize{rst_tutorials/command_line_guide:of-history-nodal-vel-id-x1-y1}]{\sphinxcrossref{of.history.nodal.vel id x1 y1}}}

\item {} 
\phantomsection\label{\detokenize{rst_tutorials/command_line_guide:id72}}{\hyperref[\detokenize{rst_tutorials/command_line_guide:of-history-nodal-dis-id-x1-y1}]{\sphinxcrossref{of.history.nodal.dis id x1 y1}}}

\item {} 
\phantomsection\label{\detokenize{rst_tutorials/command_line_guide:id73}}{\hyperref[\detokenize{rst_tutorials/command_line_guide:of-history-nodal-fluid-pressure-id-x1-y1}]{\sphinxcrossref{of.history.nodal.fluid.pressure id x1 y1}}}

\item {} 
\phantomsection\label{\detokenize{rst_tutorials/command_line_guide:id74}}{\hyperref[\detokenize{rst_tutorials/command_line_guide:of-history-nodal-fracture-pressure-id-x1-y1}]{\sphinxcrossref{of.history.nodal.fracture.pressure id x1 y1}}}

\item {} 
\phantomsection\label{\detokenize{rst_tutorials/command_line_guide:id75}}{\hyperref[\detokenize{rst_tutorials/command_line_guide:of-history-nodal-matrix-pressure-id-x1-y1}]{\sphinxcrossref{of.history.nodal.matrix.pressure id x1 y1}}}

\item {} 
\phantomsection\label{\detokenize{rst_tutorials/command_line_guide:id76}}{\hyperref[\detokenize{rst_tutorials/command_line_guide:of-history-nodal-temperature-id-x1-y1}]{\sphinxcrossref{of.history.nodal.temperature id x1 y1}}}

\item {} 
\phantomsection\label{\detokenize{rst_tutorials/command_line_guide:id77}}{\hyperref[\detokenize{rst_tutorials/command_line_guide:of-history-nodal-group-force-id-groupname}]{\sphinxcrossref{of.history.nodal.group.force id groupname}}}

\item {} 
\phantomsection\label{\detokenize{rst_tutorials/command_line_guide:id78}}{\hyperref[\detokenize{rst_tutorials/command_line_guide:of-history-nodal-group-vel-id-groupname}]{\sphinxcrossref{of.history.nodal.group.vel id groupname}}}

\item {} 
\phantomsection\label{\detokenize{rst_tutorials/command_line_guide:id79}}{\hyperref[\detokenize{rst_tutorials/command_line_guide:of-history-nodal-group-dis-id-groupname}]{\sphinxcrossref{of.history.nodal.group.dis id groupname}}}

\item {} 
\phantomsection\label{\detokenize{rst_tutorials/command_line_guide:id80}}{\hyperref[\detokenize{rst_tutorials/command_line_guide:of-history-nodal-group-fluid-pressure-id-groupname}]{\sphinxcrossref{of.history.nodal.group.fluid.pressure id groupname}}}

\item {} 
\phantomsection\label{\detokenize{rst_tutorials/command_line_guide:id81}}{\hyperref[\detokenize{rst_tutorials/command_line_guide:of-history-nodal-group-fracture-pressure-id-groupname}]{\sphinxcrossref{of.history.nodal.group.fracture.pressure id groupname}}}

\item {} 
\phantomsection\label{\detokenize{rst_tutorials/command_line_guide:id82}}{\hyperref[\detokenize{rst_tutorials/command_line_guide:of-history-nodal-group-matrix-pressure-id-groupname}]{\sphinxcrossref{of.history.nodal.group.matrix.pressure id groupname}}}

\item {} 
\phantomsection\label{\detokenize{rst_tutorials/command_line_guide:id83}}{\hyperref[\detokenize{rst_tutorials/command_line_guide:of-history-nodal-group-temperature-id-groupname}]{\sphinxcrossref{of.history.nodal.group.temperature id groupname}}}

\item {} 
\phantomsection\label{\detokenize{rst_tutorials/command_line_guide:id84}}{\hyperref[\detokenize{rst_tutorials/command_line_guide:of-history-element-stress-id-x1-y1}]{\sphinxcrossref{of.history.element.stress id x1 y1}}}

\item {} 
\phantomsection\label{\detokenize{rst_tutorials/command_line_guide:id85}}{\hyperref[\detokenize{rst_tutorials/command_line_guide:of-history-element-strain-id-x1-y1}]{\sphinxcrossref{of.history.element.strain id x1 y1}}}

\item {} 
\phantomsection\label{\detokenize{rst_tutorials/command_line_guide:id86}}{\hyperref[\detokenize{rst_tutorials/command_line_guide:of-history-element-strainrate-id-x1-y1}]{\sphinxcrossref{of.history.element.strainrate id x1 y1}}}

\item {} 
\phantomsection\label{\detokenize{rst_tutorials/command_line_guide:id87}}{\hyperref[\detokenize{rst_tutorials/command_line_guide:of-history-element-group-stress-id-groupname}]{\sphinxcrossref{of.history.element.group.stress id groupname}}}

\item {} 
\phantomsection\label{\detokenize{rst_tutorials/command_line_guide:id88}}{\hyperref[\detokenize{rst_tutorials/command_line_guide:of-history-element-group-strain-id-groupname}]{\sphinxcrossref{of.history.element.group.strain id groupname}}}

\item {} 
\phantomsection\label{\detokenize{rst_tutorials/command_line_guide:id89}}{\hyperref[\detokenize{rst_tutorials/command_line_guide:of-history-element-group-strainrate-id-groupname}]{\sphinxcrossref{of.history.element.group.strainrate id groupname}}}

\item {} 
\phantomsection\label{\detokenize{rst_tutorials/command_line_guide:id90}}{\hyperref[\detokenize{rst_tutorials/command_line_guide:of-history-cohelement-dis-id-x1-y1}]{\sphinxcrossref{of.history.cohelement.dis id x1 y1}}}

\item {} 
\phantomsection\label{\detokenize{rst_tutorials/command_line_guide:id91}}{\hyperref[\detokenize{rst_tutorials/command_line_guide:of-history-cohelement-force-id-x1-y1}]{\sphinxcrossref{of.history.cohelement.force id x1 y1}}}

\item {} 
\phantomsection\label{\detokenize{rst_tutorials/command_line_guide:id92}}{\hyperref[\detokenize{rst_tutorials/command_line_guide:of-history-cohelement-vel-id-x1-y1}]{\sphinxcrossref{of.history.cohelement.vel id x1 y1}}}

\item {} 
\phantomsection\label{\detokenize{rst_tutorials/command_line_guide:id93}}{\hyperref[\detokenize{rst_tutorials/command_line_guide:of-history-cohelement-shearstrength-id-x1-y1}]{\sphinxcrossref{of.history.cohelement.shearstrength id x1 y1}}}

\item {} 
\phantomsection\label{\detokenize{rst_tutorials/command_line_guide:id94}}{\hyperref[\detokenize{rst_tutorials/command_line_guide:of-history-cohelement-group-dis-id-groupname}]{\sphinxcrossref{of.history.cohelement.group.dis id groupname}}}

\item {} 
\phantomsection\label{\detokenize{rst_tutorials/command_line_guide:id95}}{\hyperref[\detokenize{rst_tutorials/command_line_guide:of-history-cohelement-group-force-id-groupname}]{\sphinxcrossref{of.history.cohelement.group.force id groupname}}}

\item {} 
\phantomsection\label{\detokenize{rst_tutorials/command_line_guide:id96}}{\hyperref[\detokenize{rst_tutorials/command_line_guide:of-history-cohelement-group-vel-id-groupname}]{\sphinxcrossref{of.history.cohelement.group.vel id groupname}}}

\item {} 
\phantomsection\label{\detokenize{rst_tutorials/command_line_guide:id97}}{\hyperref[\detokenize{rst_tutorials/command_line_guide:of-history-cohelement-group-shearstrength-id-groupname}]{\sphinxcrossref{of.history.cohelement.group.shearstrength id groupname}}}

\item {} 
\phantomsection\label{\detokenize{rst_tutorials/command_line_guide:id98}}{\hyperref[\detokenize{rst_tutorials/command_line_guide:of-history-energy}]{\sphinxcrossref{of.history.energy}}}

\item {} 
\phantomsection\label{\detokenize{rst_tutorials/command_line_guide:id99}}{\hyperref[\detokenize{rst_tutorials/command_line_guide:of-history-unbalance}]{\sphinxcrossref{of.history.unbalance}}}

\item {} 
\phantomsection\label{\detokenize{rst_tutorials/command_line_guide:id100}}{\hyperref[\detokenize{rst_tutorials/command_line_guide:of-history-solveratio}]{\sphinxcrossref{of.history.solveratio}}}

\item {} 
\phantomsection\label{\detokenize{rst_tutorials/command_line_guide:id101}}{\hyperref[\detokenize{rst_tutorials/command_line_guide:of-history-interval-intervalvalue}]{\sphinxcrossref{of.history.interval intervalvalue}}}

\item {} 
\phantomsection\label{\detokenize{rst_tutorials/command_line_guide:id102}}{\hyperref[\detokenize{rst_tutorials/command_line_guide:of-history-pv-interval-intervalvalue}]{\sphinxcrossref{of.history.pv.interval intervalvalue}}}

\item {} 
\phantomsection\label{\detokenize{rst_tutorials/command_line_guide:id103}}{\hyperref[\detokenize{rst_tutorials/command_line_guide:of-history-pv-reduced-interval-intervalvalue-fracturethreshold-intervalvalue}]{\sphinxcrossref{of.history.pv.reduced.interval intervalvalue fracturethreshold intervalvalue}}}

\item {} 
\phantomsection\label{\detokenize{rst_tutorials/command_line_guide:id104}}{\hyperref[\detokenize{rst_tutorials/command_line_guide:paraview}]{\sphinxcrossref{paraview}}}

\end{itemize}

\item {} 
\phantomsection\label{\detokenize{rst_tutorials/command_line_guide:id105}}{\hyperref[\detokenize{rst_tutorials/command_line_guide:dfn}]{\sphinxcrossref{dfn}}}
\begin{itemize}
\item {} 
\phantomsection\label{\detokenize{rst_tutorials/command_line_guide:id106}}{\hyperref[\detokenize{rst_tutorials/command_line_guide:of-dfn-connectivity-dfnnum}]{\sphinxcrossref{of.dfn.connectivity dfnnum}}}

\end{itemize}

\item {} 
\phantomsection\label{\detokenize{rst_tutorials/command_line_guide:id107}}{\hyperref[\detokenize{rst_tutorials/command_line_guide:table}]{\sphinxcrossref{table}}}
\begin{itemize}
\item {} 
\phantomsection\label{\detokenize{rst_tutorials/command_line_guide:id108}}{\hyperref[\detokenize{rst_tutorials/command_line_guide:of-table-tablename-table1-dat}]{\sphinxcrossref{of.table tablename ‘table1.dat’}}}

\end{itemize}

\item {} 
\phantomsection\label{\detokenize{rst_tutorials/command_line_guide:id109}}{\hyperref[\detokenize{rst_tutorials/command_line_guide:damping}]{\sphinxcrossref{damping}}}
\begin{itemize}
\item {} 
\phantomsection\label{\detokenize{rst_tutorials/command_line_guide:id110}}{\hyperref[\detokenize{rst_tutorials/command_line_guide:of-damp-global-value}]{\sphinxcrossref{of.damp.global value}}}

\end{itemize}

\item {} 
\phantomsection\label{\detokenize{rst_tutorials/command_line_guide:id111}}{\hyperref[\detokenize{rst_tutorials/command_line_guide:id5}]{\sphinxcrossref{hydro}}}
\begin{itemize}
\item {} 
\phantomsection\label{\detokenize{rst_tutorials/command_line_guide:id112}}{\hyperref[\detokenize{rst_tutorials/command_line_guide:of-hydro-timestep-value}]{\sphinxcrossref{of.hydro.timestep value}}}

\end{itemize}

\item {} 
\phantomsection\label{\detokenize{rst_tutorials/command_line_guide:id113}}{\hyperref[\detokenize{rst_tutorials/command_line_guide:gravity}]{\sphinxcrossref{gravity}}}
\begin{itemize}
\item {} 
\phantomsection\label{\detokenize{rst_tutorials/command_line_guide:id114}}{\hyperref[\detokenize{rst_tutorials/command_line_guide:of-gravity-x-y}]{\sphinxcrossref{of.gravity x y}}}

\end{itemize}

\item {} 
\phantomsection\label{\detokenize{rst_tutorials/command_line_guide:id115}}{\hyperref[\detokenize{rst_tutorials/command_line_guide:seismic}]{\sphinxcrossref{seismic}}}
\begin{itemize}
\item {} 
\phantomsection\label{\detokenize{rst_tutorials/command_line_guide:id116}}{\hyperref[\detokenize{rst_tutorials/command_line_guide:of-seismic-window-value}]{\sphinxcrossref{of.seismic.window value}}}

\end{itemize}

\end{itemize}
\end{sphinxShadowBox}


\chapter{general commands}
\label{\detokenize{rst_tutorials/command_line_guide:general-commands}}\label{\detokenize{rst_tutorials/command_line_guide::doc}}

\section{of.new}
\label{\detokenize{rst_tutorials/command_line_guide:of-new}}
\sphinxstylestrong{meaning} {[}Optional{]}: Clear all variables from your last computation (i.e. start of a new OpenFDEM run).

\sphinxstylestrong{demo}:

\begin{sphinxVerbatim}[commandchars=\\\{\}]
\PYG{n}{of}\PYG{o}{.}\PYG{n}{new}
\end{sphinxVerbatim}


\section{of.restore}
\label{\detokenize{rst_tutorials/command_line_guide:of-restore}}
\sphinxstylestrong{meaning} {[}Optional{]}: call the saved binary computation results.

\sphinxstylestrong{demo}:

\begin{sphinxVerbatim}[commandchars=\\\{\}]
of.restore “filename.sav”
\end{sphinxVerbatim}


\section{of.result}
\label{\detokenize{rst_tutorials/command_line_guide:of-result}}
\sphinxstylestrong{meaning} {[}Optional{]}: assign the results folder, default value is \sphinxcode{\sphinxupquote{../result}}.

\sphinxstylestrong{demo}:

\begin{sphinxVerbatim}[commandchars=\\\{\}]
of.result “c:/user/openfdem example/result”
\end{sphinxVerbatim}


\section{of.save}
\label{\detokenize{rst_tutorials/command_line_guide:of-save}}
\sphinxstylestrong{meaning}: save the binary computation results.

\sphinxstylestrong{demo}:

\begin{sphinxVerbatim}[commandchars=\\\{\}]
of.save “c:/user/openfdem example/result/insitu.sav”
\end{sphinxVerbatim}


\section{of.import}
\label{\detokenize{rst_tutorials/command_line_guide:of-import}}
\sphinxstylestrong{meaning}: import files, currently supports all formats of \sphinxcode{\sphinxupquote{.inp}} files, \sphinxcode{\sphinxupquote{.msh}} files (no more than 2.0
version), \sphinxcode{\sphinxupquote{.msh2}} files, \sphinxcode{\sphinxupquote{.geo}} files and \sphinxcode{\sphinxupquote{.ofdem}} files (default format for OpenFDEM).

OpenFDEM read all information from your mesh, including nodal physical, group, element physical group.

OpenFDEM 4.6+ version can read \sphinxcode{\sphinxupquote{.tess}} and \sphinxcode{\sphinxupquote{.stl}}

\sphinxstylestrong{demo}:

\begin{sphinxVerbatim}[commandchars=\\\{\}]
of.import “c:/user/openfdem example/result/ucs.msh”
\end{sphinxVerbatim}


\section{of.debug on}
\label{\detokenize{rst_tutorials/command_line_guide:of-debug-on}}
\sphinxstylestrong{meaning} {[}Optional{]}: turn on the debug mode, the default is off. Debugging will increase the run\sphinxhyphen{}time but a \sphinxcode{\sphinxupquote{.log}} file will
be generated when this mode is on.

\sphinxstylestrong{demo}:

\begin{sphinxVerbatim}[commandchars=\\\{\}]
\PYG{n}{of}\PYG{o}{.}\PYG{n}{debug} \PYG{n}{on} \PYG{o+ow}{or} \PYG{n}{of}\PYG{o}{.}\PYG{n}{debug} \PYG{n}{off}
\end{sphinxVerbatim}


\chapter{config}
\label{\detokenize{rst_tutorials/command_line_guide:config}}

\section{of.config.type planestress/planestrain}
\label{\detokenize{rst_tutorials/command_line_guide:of-config-type-planestress-planestrain}}
\sphinxstylestrong{meaning}: choosing plane stress or plane strain, the default is plane stress

\sphinxstylestrong{demo}:

\begin{sphinxVerbatim}[commandchars=\\\{\}]
\PYG{n}{of}\PYG{o}{.}\PYG{n}{config}\PYG{o}{.}\PYG{n}{type} \PYG{n}{planestrain}
\end{sphinxVerbatim}


\section{of.config.module hydro/blast/thermal}
\label{\detokenize{rst_tutorials/command_line_guide:of-config-module-hydro-blast-thermal}}
\sphinxstylestrong{meaning}: choosing mechanical module, the default is mechanical

\sphinxstylestrong{demo}:

\begin{sphinxVerbatim}[commandchars=\\\{\}]
\PYG{n}{of}\PYG{o}{.}\PYG{n}{config}\PYG{o}{.}\PYG{n}{module} \PYG{n}{hydro}
\PYG{n}{of}\PYG{o}{.}\PYG{n}{config}\PYG{o}{.}\PYG{n}{module} \PYG{n}{blast}
\PYG{n}{of}\PYG{o}{.}\PYG{n}{config}\PYG{o}{.}\PYG{n}{module} \PYG{n}{thermal}
\end{sphinxVerbatim}


\chapter{geometry}
\label{\detokenize{rst_tutorials/command_line_guide:geometry}}
You can create meshes based on the built\sphinxhyphen{}in command line functions. OpenFDEM also supports creating \sphinxcode{\sphinxupquote{retangular}}, \sphinxcode{\sphinxupquote{circle}}, \sphinxcode{\sphinxupquote{ellipse}},
\sphinxcode{\sphinxupquote{arbitary polygon}} from tables.  The bool operators support \sphinxcode{\sphinxupquote{intersection}}, \sphinxcode{\sphinxupquote{union}}, \sphinxcode{\sphinxupquote{difference}}, and \sphinxcode{\sphinxupquote{fragment}} between different entities.


\section{of.geometry.square name x0 y0 x1 y1 x2 y2 x3 y3}
\label{\detokenize{rst_tutorials/command_line_guide:of-geometry-square-name-x0-y0-x1-y1-x2-y2-x3-y3}}
\sphinxstylestrong{meaning}: create a geometry using square range, and the geometry tag
is ‘name’

\sphinxstylestrong{demo}:

\begin{sphinxVerbatim}[commandchars=\\\{\}]
of.geometry.square ‘rock1’ 0.0 1.0 1.0 2.0 2.0 3.0 4.0 2.0
\end{sphinxVerbatim}


\section{of.geometry.polygon name n x0 y0 x1 y1 x2 y2 x3 y3}
\label{\detokenize{rst_tutorials/command_line_guide:of-geometry-polygon-name-n-x0-y0-x1-y1-x2-y2-x3-y3}}
\sphinxstylestrong{meaning}: create a geometry using polygon range, the geometry tag is
‘name’ and the vertex count is n

\sphinxstylestrong{demo}:

\begin{sphinxVerbatim}[commandchars=\\\{\}]
of.geometry.polygon ‘rock1’ 5 0.0 1.0 1.0 2.0 2.0 3.0 4.0 2.0 3.0 5.2
\end{sphinxVerbatim}


\section{of.geometry.table name ‘table.tab’}
\label{\detokenize{rst_tutorials/command_line_guide:of-geometry-table-name-table-tab}}
\sphinxstylestrong{meaning}: create a geometry. All point cords are put in the
table file.
\begin{description}
\item[{\sphinxstylestrong{table format}:}] \leavevmode
x0, y0
x1, y1
x2, y2
…

\end{description}

\sphinxstylestrong{demo}:

\begin{sphinxVerbatim}[commandchars=\\\{\}]
of.geometry.table ‘rock1’ ‘tablerock.tab’
\end{sphinxVerbatim}


\section{of.geometry.circle name x0 y0 r n}
\label{\detokenize{rst_tutorials/command_line_guide:of-geometry-circle-name-x0-y0-r-n}}
\sphinxstylestrong{meaning}: create a geometry using circle range, the geometry tag is
‘name’. The centric point is (x1, y1), radius is r and edge count is n, deafault is 50.

\sphinxstylestrong{demo}:

\begin{sphinxVerbatim}[commandchars=\\\{\}]
of.geometry.circle ‘tunnel’ 0.0 0.0 1.0 20
\end{sphinxVerbatim}


\section{of.geometry.ellipse name x0 y0 rmax rmin theta n}
\label{\detokenize{rst_tutorials/command_line_guide:of-geometry-ellipse-name-x0-y0-rmax-rmin-theta-n}}
\sphinxstylestrong{meaning}: create a geometry using ellipse range, where the geometry tag is
‘name’ and the centric point is (x0, y0). Radiuses are rmax and rmin,
and edge count is n, default is 50.

\sphinxstylestrong{demo}:

\begin{sphinxVerbatim}[commandchars=\\\{\}]
of.geometry.ellipse ‘ellipsetunnel’ 0.0 0.0 1.0 0.5 20
\end{sphinxVerbatim}


\section{of.geometry.cut.square name name2 x0 y0 x1 y1 x2 y2 x3 y3}
\label{\detokenize{rst_tutorials/command_line_guide:of-geometry-cut-square-name-name2-x0-y0-x1-y1-x2-y2-x3-y3}}
\sphinxstylestrong{meaning}: cut a rectangular tunnel in existing \sphinxstyleemphasis{‘name2’} group geometry,
but the \sphinxstyleemphasis{‘name’} group is not excavated.

\sphinxstylestrong{demo}:

\begin{sphinxVerbatim}[commandchars=\\\{\}]
of.geometry.cut.square ‘square\PYGZus{}tunnel’ ‘rock’ 0.0 1.0 1.0 2.0 2.0 3.0 4.0 2.0
\end{sphinxVerbatim}


\section{of.geometry.cut.polygon name name2 n x0 y0 x1 y1 x2 y2 x3 y3}
\label{\detokenize{rst_tutorials/command_line_guide:of-geometry-cut-polygon-name-name2-n-x0-y0-x1-y1-x2-y2-x3-y3}}
\sphinxstylestrong{meaning}: cut a polygon tunnel in existing \sphinxstyleemphasis{‘name2’} group geometry, but
the \sphinxstyleemphasis{‘name’} group is not excavated.

\sphinxstylestrong{demo}:

\begin{sphinxVerbatim}[commandchars=\\\{\}]
of.geometry.cut.polygon ‘polygon\PYGZus{}tunnel’ ‘rock’ 4 0.0 1.0 1.0 2.0 2.0 3.0 4.0 2.0
\end{sphinxVerbatim}


\section{of.geometry.cut.table name name2 ‘table.tab’}
\label{\detokenize{rst_tutorials/command_line_guide:of-geometry-cut-table-name-name2-table-tab}}
\sphinxstylestrong{meaning}: cut a table tunnel in existing ‘name2’ group geometry, but the
‘name’ group is not excavated.

\sphinxstylestrong{demo}:

\begin{sphinxVerbatim}[commandchars=\\\{\}]
of.geometry.cut.table ‘tunnel\PYGZus{}from\PYGZus{}table’ ‘rock’ ‘tunneltable.tab’
\end{sphinxVerbatim}


\section{of.geometry.cut.circle name name2 x0 y0 r n}
\label{\detokenize{rst_tutorials/command_line_guide:of-geometry-cut-circle-name-name2-x0-y0-r-n}}
\sphinxstylestrong{meaning}: cut a circular tunnel in existing ‘name2’ group geometry, but the
‘name’ group is not excavated.

\sphinxstylestrong{demo}:

\begin{sphinxVerbatim}[commandchars=\\\{\}]
of.geometry.cut.circle ‘circle\PYGZus{}tunnel’ ‘rock’ 0.0 0.0 1.0 20
\end{sphinxVerbatim}


\section{of.geometry.cut.ellipse name name2 x0 y0 rmax rmin theta n}
\label{\detokenize{rst_tutorials/command_line_guide:of-geometry-cut-ellipse-name-name2-x0-y0-rmax-rmin-theta-n}}
\sphinxstylestrong{meaning}: cut a elliptical tunnel in existing ‘name2’ group geometry, but
the ‘name’ group is not excavated.

\sphinxstylestrong{demo}:

\begin{sphinxVerbatim}[commandchars=\\\{\}]
of.geometry.cut.ellipse ‘ellipse\PYGZus{}tunnel’ ‘rock’ 0.0 0.0 1.0 0.5 20
\end{sphinxVerbatim}


\section{of.geometry.cut.jset name name2 dip length space gap}
\label{\detokenize{rst_tutorials/command_line_guide:of-geometry-cut-jset-name-name2-dip-length-space-gap}}
\sphinxstylestrong{meaning}: create jset ‘name2’ group geometry in ‘name’ group

method: default, normal, uniform, exponential, log\_normal, power

cdip ndip udip edip logdip pdip

dip: dip udip ndip

length: length nlength flength plength

space: space nspace

gap: gap ngap

\sphinxstylestrong{demo}:

\begin{sphinxVerbatim}[commandchars=\\\{\}]
\PYG{n}{of}\PYG{o}{.}\PYG{n}{geometry}\PYG{o}{.}\PYG{n}{cut}\PYG{o}{.}\PYG{n}{jset} \PYG{l+s+s1}{\PYGZsq{}}\PYG{l+s+s1}{jset\PYGZhy{}1}\PYG{l+s+s1}{\PYGZsq{}} \PYG{l+s+s1}{\PYGZsq{}}\PYG{l+s+s1}{rock\PYGZus{}0}\PYG{l+s+s1}{\PYGZsq{}} \PYG{n}{n\PYGZus{}dip} \PYG{l+m+mi}{60} \PYG{l+m+mi}{20} \PYG{n}{n\PYGZus{}space} \PYG{l+m+mf}{0.2} \PYG{l+m+mf}{0.1}
\end{sphinxVerbatim}


\section{of.geometry.cut.dfn name name2 method dip length}
\label{\detokenize{rst_tutorials/command_line_guide:of-geometry-cut-dfn-name-name2-method-dip-length}}
method: count, p21(length/area), p10 (n/edge length)

dip: dip udip ndip

length: length nlength flength

\sphinxstylestrong{demo}:

\begin{sphinxVerbatim}[commandchars=\\\{\}]
\PYG{n}{of}\PYG{o}{.}\PYG{n}{geometry}\PYG{o}{.}\PYG{n}{cut}\PYG{o}{.}\PYG{n}{dfn} \PYG{l+s+s1}{\PYGZsq{}}\PYG{l+s+s1}{dfn\PYGZhy{}1}\PYG{l+s+s1}{\PYGZsq{}} \PYG{l+s+s1}{\PYGZsq{}}\PYG{l+s+s1}{rock\PYGZus{}0}\PYG{l+s+s1}{\PYGZsq{}} \PYG{n}{p10} \PYG{l+m+mi}{3} \PYG{n}{n\PYGZus{}dip} \PYG{l+m+mi}{60} \PYG{l+m+mi}{5} \PYG{n}{n\PYGZus{}length} \PYG{l+m+mf}{0.3} \PYG{l+m+mf}{0.1}
\end{sphinxVerbatim}


\section{of.geometry.import.rdfn pixel\_ratio name name2 name}
\label{\detokenize{rst_tutorials/command_line_guide:of-geometry-import-rdfn-pixel-ratio-name-name2-name}}
\sphinxstylestrong{demo}:

\begin{sphinxVerbatim}[commandchars=\\\{\}]
\PYG{n}{of}\PYG{o}{.}\PYG{n}{geometry}\PYG{o}{.}\PYG{n}{import}\PYG{o}{.}\PYG{n}{rdfn} \PYG{l+m+mf}{1.0} \PYG{l+s+s1}{\PYGZsq{}}\PYG{l+s+s1}{dfn\PYGZhy{}1}\PYG{l+s+s1}{\PYGZsq{}} \PYG{l+s+s1}{\PYGZsq{}}\PYG{l+s+s1}{rock\PYGZus{}0}\PYG{l+s+s1}{\PYGZsq{}} \PYG{l+s+s1}{\PYGZsq{}}\PYG{l+s+s1}{fault.dat}\PYG{l+s+s1}{\PYGZsq{}}
\end{sphinxVerbatim}


\section{of.geometry.remove.square name name2 x0 y0 x1 y1 x2 y2 x3 y3}
\label{\detokenize{rst_tutorials/command_line_guide:of-geometry-remove-square-name-name2-x0-y0-x1-y1-x2-y2-x3-y3}}
\sphinxstylestrong{meaning}: remove a rectangular tunnel in existing ‘name2’ group geometry,
the ‘name’ group is excavated as null

\sphinxstylestrong{demo}:

\begin{sphinxVerbatim}[commandchars=\\\{\}]
of.geometry.remove.square ‘rock2’ ‘rock’ 0.0 1.0 1.0 2.0 2.0 3.0 4.0 2.0
\end{sphinxVerbatim}


\section{of.geometry.remove.polygon name name2 n x0 y0 x1 y1 x2 y2 x3 y3}
\label{\detokenize{rst_tutorials/command_line_guide:of-geometry-remove-polygon-name-name2-n-x0-y0-x1-y1-x2-y2-x3-y3}}
\sphinxstylestrong{meaning}: remove a polygon tunnel in existing ‘name2’ group geometry, the
‘name’ group is excavated as null

\sphinxstylestrong{demo}:

\begin{sphinxVerbatim}[commandchars=\\\{\}]
of.geometry.remove.polygon ‘rock2’ ‘rock’ 0.0 1.0 1.0 2.0 2.0 3.0 4.0 2.0
\end{sphinxVerbatim}


\section{of.geometry.remove.table name name2 ‘table.tab’}
\label{\detokenize{rst_tutorials/command_line_guide:of-geometry-remove-table-name-name2-table-tab}}
\sphinxstylestrong{meaning}: remove a table tunnel in existing ‘name2’ group geometry, the
‘name’ group is excavated as null

\sphinxstylestrong{demo}:

\begin{sphinxVerbatim}[commandchars=\\\{\}]
of.geometry.remove.table ‘rock2’ ‘rock’ 0.0 1.0 1.0 2.0 2.0 3.0 4.0 2.0
\end{sphinxVerbatim}


\section{of.geometry.remove.circle name name2 x1 y1 r n}
\label{\detokenize{rst_tutorials/command_line_guide:of-geometry-remove-circle-name-name2-x1-y1-r-n}}
\sphinxstylestrong{meaning}: remove a circle tunnel in existing ‘name2’ group geometry, the
‘name’ group is excavated as null

\sphinxstylestrong{demo}:

\begin{sphinxVerbatim}[commandchars=\\\{\}]
of.geometry.remove.circle ‘rock2’ ‘rock’ 0.0 1.0 1.0 2.0 2.0 3.0 4.0 2.0
\end{sphinxVerbatim}


\section{of.geometry.remove.ellipse name name2 x0 y0 rmax rmin theta n}
\label{\detokenize{rst_tutorials/command_line_guide:of-geometry-remove-ellipse-name-name2-x0-y0-rmax-rmin-theta-n}}
\sphinxstylestrong{meaning}: remove a ellipse tunnel in existing ‘name2’ group geometry, the
‘name’ group is excavated as null

\sphinxstylestrong{demo}:

\begin{sphinxVerbatim}[commandchars=\\\{\}]
of.geometry.remove.ellipse ‘rock2’ ‘rock’ 0.0 1.0 1.0 2.0 2.0 3.0 4.0 2.0
\end{sphinxVerbatim}


\section{of.geometry.mesh.size name default}
\label{\detokenize{rst_tutorials/command_line_guide:of-geometry-mesh-size-name-default}}
\sphinxstylestrong{meaning}: assign the mesh size to corresponding geometry groups, ‘all’
means the all geometries

\sphinxstylestrong{demo}:

\begin{sphinxVerbatim}[commandchars=\\\{\}]
\PYG{n}{of}\PYG{o}{.}\PYG{n}{geometry}\PYG{o}{.}\PYG{n}{mesh}\PYG{o}{.}\PYG{n}{size} \PYG{l+s+s1}{\PYGZsq{}}\PYG{l+s+s1}{all}\PYG{l+s+s1}{\PYGZsq{}} \PYG{l+m+mf}{0.2}
\end{sphinxVerbatim}


\section{of.geometry.mesh keyword}
\label{\detokenize{rst_tutorials/command_line_guide:of-geometry-mesh-keyword}}
\sphinxstylestrong{meaning}: method used to generate mesh, default is delaunay

keyword: meshadapt(default), delaunay, frontal\sphinxhyphen{}delaunay

\sphinxstylestrong{demo}:

\begin{sphinxVerbatim}[commandchars=\\\{\}]
\PYG{n}{of}\PYG{o}{.}\PYG{n}{geometry}\PYG{o}{.}\PYG{n}{mesh} \PYG{n}{meshasapt}
\end{sphinxVerbatim}


\chapter{mesh\_insert}
\label{\detokenize{rst_tutorials/command_line_guide:mesh-insert}}

\section{of.mesh.insert elementgroup}
\label{\detokenize{rst_tutorials/command_line_guide:of-mesh-insert-elementgroup}}
\sphinxstylestrong{meaning}: insert the CZMs for specific element groups, ‘all’ means insert
CZMs in all model

\sphinxstylestrong{demo}:

\begin{sphinxVerbatim}[commandchars=\\\{\}]
of.mesh.insert ‘rock’

of.mesh.insert ‘all’
\end{sphinxVerbatim}


\chapter{nodal}
\label{\detokenize{rst_tutorials/command_line_guide:nodal}}

\section{of.nodal.coord number\_nodes}
\label{\detokenize{rst_tutorials/command_line_guide:of-nodal-coord-number-nodes}}
\sphinxstylestrong{meaning}: input nodes for the .fdem format

\sphinxstylestrong{demo}:

\begin{sphinxVerbatim}[commandchars=\\\{\}]
\PYG{n}{of}\PYG{o}{.}\PYG{n}{nodal}\PYG{o}{.}\PYG{n}{coord} \PYG{l+m+mi}{3}

  \PYG{l+m+mf}{0.1} \PYG{l+m+mf}{0.3}

  \PYG{l+m+mf}{0.2} \PYG{l+m+mf}{0.6}

  \PYG{l+m+mf}{0.8} \PYG{l+m+mf}{0.2}
\end{sphinxVerbatim}


\chapter{element}
\label{\detokenize{rst_tutorials/command_line_guide:element}}

\section{of.element.connectivity number\_nodes number\_elements}
\label{\detokenize{rst_tutorials/command_line_guide:of-element-connectivity-number-nodes-number-elements}}\begin{quote}

x1 y1 x2 y2 x3 y3…
\end{quote}

\sphinxstylestrong{meaning}: input elements for the .fdem format, in anticlockwise order

\sphinxstylestrong{demo}:

\begin{sphinxVerbatim}[commandchars=\\\{\}]
\PYG{n}{of}\PYG{o}{.}\PYG{n}{element}\PYG{o}{.}\PYG{n}{connectivity} \PYG{l+m+mi}{3} \PYG{l+m+mi}{3}

  \PYG{l+m+mi}{0} \PYG{l+m+mi}{3} \PYG{l+m+mi}{15}

  \PYG{l+m+mi}{1} \PYG{l+m+mi}{2} \PYG{l+m+mi}{7}

  \PYG{l+m+mi}{2} \PYG{l+m+mi}{4} \PYG{l+m+mi}{6}
\end{sphinxVerbatim}


\chapter{contact}
\label{\detokenize{rst_tutorials/command_line_guide:contact}}

\section{of.contact.detection nbs}
\label{\detokenize{rst_tutorials/command_line_guide:of-contact-detection-nbs}}
\sphinxstylestrong{meaning}: the contact detection method, current methods are: Munjiza model, NBS,
modified NBS and Li\sphinxhyphen{}Grasselli’s method {[}default{]}.

\sphinxstylestrong{demo}:

\begin{sphinxVerbatim}[commandchars=\\\{\}]
\PYG{n}{of}\PYG{o}{.}\PYG{n}{contact}\PYG{o}{.} \PYG{n}{detection} \PYG{n}{nbs}
\end{sphinxVerbatim}


\section{of.contact.force lig}
\label{\detokenize{rst_tutorials/command_line_guide:of-contact-force-lig}}
\sphinxstylestrong{meaning}: the contact force method, current methods are: Munjiza model,
and Li\sphinxhyphen{}Grasselli’s method {[}default{]}.

\sphinxstylestrong{demo}:

\begin{sphinxVerbatim}[commandchars=\\\{\}]
\PYG{n}{of}\PYG{o}{.}\PYG{n}{contact}\PYG{o}{.}\PYG{n}{force} \PYG{n}{lig}
\end{sphinxVerbatim}


\chapter{cohelement}
\label{\detokenize{rst_tutorials/command_line_guide:cohelement}}

\section{of.cohelement.delete cohelementgroup}
\label{\detokenize{rst_tutorials/command_line_guide:of-cohelement-delete-cohelementgroup}}

\chapter{group}
\label{\detokenize{rst_tutorials/command_line_guide:group}}

\section{nodal groups}
\label{\detokenize{rst_tutorials/command_line_guide:nodal-groups}}

\subsection{of.group.nodal.square nodalgroupname x1 x2 y1 y2}
\label{\detokenize{rst_tutorials/command_line_guide:of-group-nodal-square-nodalgroupname-x1-x2-y1-y2}}
\sphinxstylestrong{meaning}: create nodal group using rectangular range

\sphinxstylestrong{demo}:

\begin{sphinxVerbatim}[commandchars=\\\{\}]
of.group.nodal.square ‘rock’ \PYGZhy{}50.0 50.0 \PYGZhy{}50.0 10.0
\end{sphinxVerbatim}


\subsection{of.group.nodal.circle nodalgroupname x1 y1 r}
\label{\detokenize{rst_tutorials/command_line_guide:of-group-nodal-circle-nodalgroupname-x1-y1-r}}
\sphinxstylestrong{meaning}: create nodal group using circle range, the nodes within the
circle

\sphinxstylestrong{demo}:

\begin{sphinxVerbatim}[commandchars=\\\{\}]
of.group.nodal.circle ‘tunnel’ 0.0 0.0 2.5
\end{sphinxVerbatim}


\subsection{of.group.nodal.circle.outer nodalgroupname x1 y1 r}
\label{\detokenize{rst_tutorials/command_line_guide:of-group-nodal-circle-outer-nodalgroupname-x1-y1-r}}
\sphinxstylestrong{meaning}: create nodal group out of the circle range

\sphinxstylestrong{demo}:

\begin{sphinxVerbatim}[commandchars=\\\{\}]
of.group.nodal.circle.outer ‘tunnel’ 0.0 0.0 2.5
\end{sphinxVerbatim}


\subsection{of.group.nodal.circle.on nodalgroupname x1 y1 r}
\label{\detokenize{rst_tutorials/command_line_guide:of-group-nodal-circle-on-nodalgroupname-x1-y1-r}}
\sphinxstylestrong{meaning}: create nodal group on of the circle boundary

\sphinxstylestrong{demo}:

\begin{sphinxVerbatim}[commandchars=\\\{\}]
of.group.nodal.circle.on ‘tunnel’ 0.0 0.0 2.5
\end{sphinxVerbatim}


\subsection{of.group.nodal.plane.left nodalgroupname x1 y1 x2 y2}
\label{\detokenize{rst_tutorials/command_line_guide:of-group-nodal-plane-left-nodalgroupname-x1-y1-x2-y2}}
\sphinxstylestrong{meaning}: create nodal group on the left of the predefined plane

\sphinxstylestrong{demo}:

\begin{sphinxVerbatim}[commandchars=\\\{\}]
of.group.nodal.plane.left ‘slope’ 0.0 0.0 1.0 2.7
\end{sphinxVerbatim}


\subsection{of.group.nodal.plane.right nodalgroupname x1 y1 x2 y2}
\label{\detokenize{rst_tutorials/command_line_guide:of-group-nodal-plane-right-nodalgroupname-x1-y1-x2-y2}}
\sphinxstylestrong{meaning}: create nodal group on the right of the plane

\sphinxstylestrong{demo}:

\begin{sphinxVerbatim}[commandchars=\\\{\}]
of.group.nodal.plane.right ‘slope’ 0.0 0.0 1.0 2.7
\end{sphinxVerbatim}


\subsection{of.group.nodal.plane nodalgroupname x1 y1 x2 y2}
\label{\detokenize{rst_tutorials/command_line_guide:of-group-nodal-plane-nodalgroupname-x1-y1-x2-y2}}
\sphinxstylestrong{meaning}: create nodal group on the plane

\sphinxstylestrong{demo}:

\begin{sphinxVerbatim}[commandchars=\\\{\}]
of.group.nodal.plane ‘slope’ 0.0 0.0 1.0 2.7
\end{sphinxVerbatim}


\subsection{of.group.nodal.from.element ‘up’ ‘up’}
\label{\detokenize{rst_tutorials/command_line_guide:of-group-nodal-from-element-up-up}}
\sphinxstylestrong{meaning}: create nodal group from the element group

\sphinxstylestrong{demo}:

\begin{sphinxVerbatim}[commandchars=\\\{\}]
of.group.nodal.from.element ‘rock’ ‘rock’
\end{sphinxVerbatim}


\section{bool}
\label{\detokenize{rst_tutorials/command_line_guide:bool}}

\subsection{of.group.nodal.bool.union nodalgroupnename nodalgroupnamea nodalgroupnameb}
\label{\detokenize{rst_tutorials/command_line_guide:of-group-nodal-bool-union-nodalgroupnename-nodalgroupnamea-nodalgroupnameb}}
\sphinxstylestrong{meaning}: create nodal group using bool union

\sphinxstylestrong{demo}:

\begin{sphinxVerbatim}[commandchars=\\\{\}]
of.group.nodal.bool.union ‘rock1’ ‘rock2’
\end{sphinxVerbatim}


\subsection{of.group.nodal.bool.intersect nodalgroupnewname nodalgroupnamea nodalgroupnameb}
\label{\detokenize{rst_tutorials/command_line_guide:of-group-nodal-bool-intersect-nodalgroupnewname-nodalgroupnamea-nodalgroupnameb}}
\sphinxstylestrong{meaning}: create nodal group using bool intersect

\sphinxstylestrong{demo}:

\begin{sphinxVerbatim}[commandchars=\\\{\}]
of.group.nodal.bool.intersect ‘rock1’ ‘rock2’
\end{sphinxVerbatim}


\subsection{of.group.nodal.bool.subtract nodalgroupnewname nodalgroupnamea nodalgroupnameb a\sphinxhyphen{}b}
\label{\detokenize{rst_tutorials/command_line_guide:of-group-nodal-bool-subtract-nodalgroupnewname-nodalgroupnamea-nodalgroupnameb-a-b}}
\sphinxstylestrong{meaning}: create nodal group using bool subtract

\sphinxstylestrong{demo}:

\begin{sphinxVerbatim}[commandchars=\\\{\}]
of.group.nodal.bool.subtract‘rock1’ ‘rock2’
\end{sphinxVerbatim}


\section{element}
\label{\detokenize{rst_tutorials/command_line_guide:element-1}}\label{\detokenize{rst_tutorials/command_line_guide:id1}}

\subsection{of.group.element.square elementgroupname x1 x2 y1 y2}
\label{\detokenize{rst_tutorials/command_line_guide:of-group-element-square-elementgroupname-x1-x2-y1-y2}}
\sphinxstylestrong{demo}:

\begin{sphinxVerbatim}[commandchars=\\\{\}]
of.group.element.square ‘rock’ \PYGZhy{}50.0 50.0 \PYGZhy{}50.0 10.0
\end{sphinxVerbatim}


\subsection{of.group.element.circle elementgroupname x1 y1 r}
\label{\detokenize{rst_tutorials/command_line_guide:of-group-element-circle-elementgroupname-x1-y1-r}}
\sphinxstylestrong{demo}:

\begin{sphinxVerbatim}[commandchars=\\\{\}]
of.group.element. circle ‘tunnel’ 0.0 0.0 2.5
\end{sphinxVerbatim}


\subsection{of.group.element.circle.outer elementgroupname x1 y1 r}
\label{\detokenize{rst_tutorials/command_line_guide:of-group-element-circle-outer-elementgroupname-x1-y1-r}}

\subsection{of.group.element.plane.left elementgroupname x1 y1 x2 y2}
\label{\detokenize{rst_tutorials/command_line_guide:of-group-element-plane-left-elementgroupname-x1-y1-x2-y2}}
\sphinxstylestrong{demo}:

\begin{sphinxVerbatim}[commandchars=\\\{\}]
of.group.element. .left ‘slope’ 0.0 0.0 1.0 2.7
\end{sphinxVerbatim}


\subsection{of.group.element.plane.right elementgroupname x1 y1 x2 y2}
\label{\detokenize{rst_tutorials/command_line_guide:of-group-element-plane-right-elementgroupname-x1-y1-x2-y2}}

\subsection{of.group.element.bool.union elementgroupnewname elementgroupnamea elementgroupnameb}
\label{\detokenize{rst_tutorials/command_line_guide:of-group-element-bool-union-elementgroupnewname-elementgroupnamea-elementgroupnameb}}

\subsection{of.group. element.bool.intersect elementgroupnewname elementgroupnamea elementgroupnameb}
\label{\detokenize{rst_tutorials/command_line_guide:of-group-element-bool-intersect-elementgroupnewname-elementgroupnamea-elementgroupnameb}}

\subsection{of.group. element.bool.subtract elementgroupnewname elementgroupnamea elementgroupnameb a\sphinxhyphen{}b}
\label{\detokenize{rst_tutorials/command_line_guide:of-group-element-bool-subtract-elementgroupnewname-elementgroupnamea-elementgroupnameb-a-b}}\label{\detokenize{rst_tutorials/command_line_guide:section-1}}

\section{cohesive element}
\label{\detokenize{rst_tutorials/command_line_guide:cohesive-element}}

\subsection{of.group.cohelement.square cohelementgroupname x1 x2 y1 y2}
\label{\detokenize{rst_tutorials/command_line_guide:of-group-cohelement-square-cohelementgroupname-x1-x2-y1-y2}}
\sphinxstylestrong{demo}:

\begin{sphinxVerbatim}[commandchars=\\\{\}]
of.group. cohelement.square ‘rock’ \PYGZhy{}50.0 50.0 \PYGZhy{}50.0 10.0
\end{sphinxVerbatim}


\subsection{of.group.cohelement.circle cohelementgroupname x1 y1 r (default)}
\label{\detokenize{rst_tutorials/command_line_guide:of-group-cohelement-circle-cohelementgroupname-x1-y1-r-default}}

\subsection{of.group.cohelement.circle.outer cohelementgroupname x1 y1 r}
\label{\detokenize{rst_tutorials/command_line_guide:of-group-cohelement-circle-outer-cohelementgroupname-x1-y1-r}}\label{\detokenize{rst_tutorials/command_line_guide:section-2}}

\subsection{of.group.cohelement.circle.on cohelementgroupname x1 y1 r}
\label{\detokenize{rst_tutorials/command_line_guide:of-group-cohelement-circle-on-cohelementgroupname-x1-y1-r}}\label{\detokenize{rst_tutorials/command_line_guide:section-3}}

\subsection{of.group.cohelement.plane.left cohelementgroupname x1 y1 x2 y2 (default)}
\label{\detokenize{rst_tutorials/command_line_guide:of-group-cohelement-plane-left-cohelementgroupname-x1-y1-x2-y2-default}}\label{\detokenize{rst_tutorials/command_line_guide:section-4}}

\subsection{of.group.cohelement.plane.right cohelementgroupname x1 y1 x2 y2}
\label{\detokenize{rst_tutorials/command_line_guide:of-group-cohelement-plane-right-cohelementgroupname-x1-y1-x2-y2}}\label{\detokenize{rst_tutorials/command_line_guide:section-5}}

\subsection{of.group.cohelement.plane cohelementgroupname x1 y1 x2 y2 (default)}
\label{\detokenize{rst_tutorials/command_line_guide:of-group-cohelement-plane-cohelementgroupname-x1-y1-x2-y2-default}}\label{\detokenize{rst_tutorials/command_line_guide:section-6}}

\subsection{of.group.cohelement.gbm cohelementgroupname elementgroupname1 elementgroupname2}
\label{\detokenize{rst_tutorials/command_line_guide:of-group-cohelement-gbm-cohelementgroupname-elementgroupname1-elementgroupname2}}\label{\detokenize{rst_tutorials/command_line_guide:section-7}}
\sphinxstylestrong{demo}:

\begin{sphinxVerbatim}[commandchars=\\\{\}]
\PYG{n}{of}\PYG{o}{.}\PYG{n}{group}\PYG{o}{.}\PYG{n}{cohelement}\PYG{o}{.}\PYG{n}{gbm} \PYG{l+s+s1}{\PYGZsq{}}\PYG{l+s+s1}{qtz\PYGZhy{}qtz}\PYG{l+s+s1}{\PYGZsq{}} \PYG{l+s+s1}{\PYGZsq{}}\PYG{l+s+s1}{qtz}\PYG{l+s+s1}{\PYGZsq{}} \PYG{l+s+s1}{\PYGZsq{}}\PYG{l+s+s1}{qtz}\PYG{l+s+s1}{\PYGZsq{}} \PYG{c+c1}{\PYGZsh{} quartz group}
\end{sphinxVerbatim}


\subsection{of.group.cohelement.dfn cohelementgroupname dfnname}
\label{\detokenize{rst_tutorials/command_line_guide:of-group-cohelement-dfn-cohelementgroupname-dfnname}}
\sphinxstylestrong{demo}:

\begin{sphinxVerbatim}[commandchars=\\\{\}]
\PYG{n}{of}\PYG{o}{.}\PYG{n}{group}\PYG{o}{.}\PYG{n}{cohelement}\PYG{o}{.}\PYG{n}{dfn} \PYG{l+s+s1}{\PYGZsq{}}\PYG{l+s+s1}{dfn1}\PYG{l+s+s1}{\PYGZsq{}} \PYG{l+s+s1}{\PYGZsq{}}\PYG{l+s+s1}{dfn1}\PYG{l+s+s1}{\PYGZsq{}}
\end{sphinxVerbatim}


\subsection{of.group.cohelement.bool.union cohelementgroupnewname cohelementgroupnamea cohelementgroupnameb}
\label{\detokenize{rst_tutorials/command_line_guide:of-group-cohelement-bool-union-cohelementgroupnewname-cohelementgroupnamea-cohelementgroupnameb}}

\subsection{of.group. cohelement.bool.intersect cohelementgroupnewname cohelementgroupnamea cohelementgroupnameb}
\label{\detokenize{rst_tutorials/command_line_guide:of-group-cohelement-bool-intersect-cohelementgroupnewname-cohelementgroupnamea-cohelementgroupnameb}}\label{\detokenize{rst_tutorials/command_line_guide:section-8}}

\subsection{of.group. cohelement.bool.subtract cohelementgroupnewname cohelementgroupnamea elementgroupnameb a\sphinxhyphen{}b}
\label{\detokenize{rst_tutorials/command_line_guide:of-group-cohelement-bool-subtract-cohelementgroupnewname-cohelementgroupnamea-elementgroupnameb-a-b}}\label{\detokenize{rst_tutorials/command_line_guide:section-9}}

\chapter{boundary}
\label{\detokenize{rst_tutorials/command_line_guide:boundary}}

\section{nodal}
\label{\detokenize{rst_tutorials/command_line_guide:nodal-1}}\label{\detokenize{rst_tutorials/command_line_guide:id2}}

\subsection{of.boundary.nodal.force nodalgroupname xy force\_x force\_y}
\label{\detokenize{rst_tutorials/command_line_guide:of-boundary-nodal-force-nodalgroupname-xy-force-x-force-y}}
\sphinxstylestrong{meaning}: assign force boundary to nodal groups

\sphinxstylestrong{demo}:

\begin{sphinxVerbatim}[commandchars=\\\{\}]
of.boundary.nodal.force ‘rock’ xy \PYGZhy{}50.0 27.0
of.boundary.nodal.force ‘rock’ x \PYGZhy{}50.0
of.boundary.nodal.force ‘rock’ y 27.0
\end{sphinxVerbatim}


\subsection{of.boundary.nodal.velocity nodalgroupname xy vel\_x vel\_y}
\label{\detokenize{rst_tutorials/command_line_guide:of-boundary-nodal-velocity-nodalgroupname-xy-vel-x-vel-y}}
\sphinxstylestrong{meaning}: assign normal velocity boundary to nodal groups (in local
coords)


\subsection{of.boundary.nodal.inivelocity nodalgroupname xy vel\_x vel\_y}
\label{\detokenize{rst_tutorials/command_line_guide:of-boundary-nodal-inivelocity-nodalgroupname-xy-vel-x-vel-y}}
\sphinxstylestrong{meaning}: assign initial velocity boundary to nodal groups (in local
coords)


\subsection{of.boundary.nodal.acceleration nodalgroupname xy acc\_x acc\_y}
\label{\detokenize{rst_tutorials/command_line_guide:of-boundary-nodal-acceleration-nodalgroupname-xy-acc-x-acc-y}}
\sphinxstylestrong{meaning}: assign aceleration boundary to nodal groups (in local coords)


\subsection{of.boundary.nodal.viscous nodalgroupnamexy viscous\_x viscous\_y}
\label{\detokenize{rst_tutorials/command_line_guide:of-boundary-nodal-viscous-nodalgroupnamexy-viscous-x-viscous-y}}
\sphinxstylestrong{demo}:

\begin{sphinxVerbatim}[commandchars=\\\{\}]
of.boundary.nodal.viscous ‘rock’ xy
of.boundary.nodal.viscous ‘rock’ x
of.boundary.nodal.viscous ‘rock’ y
\end{sphinxVerbatim}


\subsection{of.boundary.nodal.viscous nodalgroupnamexy viscous\_x viscous\_y}
\label{\detokenize{rst_tutorials/command_line_guide:of-boundary-nodal-viscous-nodalgroupnamexy-viscous-x-viscous-y-1}}\label{\detokenize{rst_tutorials/command_line_guide:id3}}

\section{nodal local edge}
\label{\detokenize{rst_tutorials/command_line_guide:nodal-local-edge}}

\subsection{of.boundary.nodal.force.local nodalgroupname xy force\_normal force\_shear}
\label{\detokenize{rst_tutorials/command_line_guide:of-boundary-nodal-force-local-nodalgroupname-xy-force-normal-force-shear}}

\subsection{of.boundary.nodal.velocity.local nodalgroupname xy vel\_ normal vel\_ shear}
\label{\detokenize{rst_tutorials/command_line_guide:of-boundary-nodal-velocity-local-nodalgroupname-xy-vel-normal-vel-shear}}

\subsection{of.boundary.nodal.acceleration.local nodalgroupname xy acc\_ normal acc\_ shear}
\label{\detokenize{rst_tutorials/command_line_guide:of-boundary-nodal-acceleration-local-nodalgroupname-xy-acc-normal-acc-shear}}
\sphinxstylestrong{demo}:

\begin{sphinxVerbatim}[commandchars=\\\{\}]
of.boundary.nodal.force.local ‘rock’ xy \PYGZhy{}50.0 27.0
of.boundary.nodal.force.local ‘rock’ x \PYGZhy{}50.0
of.boundary.nodal.force.local ‘rock’ y 27.0
\end{sphinxVerbatim}


\subsection{of.boundary.nodal.force.local.table nodalgroupname xy force\_normal force\_shear tablename}
\label{\detokenize{rst_tutorials/command_line_guide:of-boundary-nodal-force-local-table-nodalgroupname-xy-force-normal-force-shear-tablename}}

\subsection{of.boundary.nodal.velocity. table nodalgroupname xy vel\_ normal vel\_ shear tablename}
\label{\detokenize{rst_tutorials/command_line_guide:of-boundary-nodal-velocity-table-nodalgroupname-xy-vel-normal-vel-shear-tablename}}

\subsection{of.boundary.nodal.acceleration. table nodalgroupname xy acc\_ normal acc\_ shear tablename}
\label{\detokenize{rst_tutorials/command_line_guide:of-boundary-nodal-acceleration-table-nodalgroupname-xy-acc-normal-acc-shear-tablename}}
\sphinxstylestrong{demo}:

\begin{sphinxVerbatim}[commandchars=\\\{\}]
of.boundary.nodal.force.local ‘rock’ xy \PYGZhy{}50.0 27.0

  of.boundary.nodal.force.local ‘rock’ x \PYGZhy{}50.0

  of.boundary.nodal.force.local ‘rock’ y 27.0
\end{sphinxVerbatim}


\section{nodal table}
\label{\detokenize{rst_tutorials/command_line_guide:nodal-table}}

\subsection{of.boundary.nodal.force.table nodalgroupname xy force\_x force\_y tablename}
\label{\detokenize{rst_tutorials/command_line_guide:of-boundary-nodal-force-table-nodalgroupname-xy-force-x-force-y-tablename}}
\sphinxstylestrong{demo}:

\begin{sphinxVerbatim}[commandchars=\\\{\}]
of.boundary.nodal.force.table ‘rock’ xy \PYGZhy{}50.0 27.0 blast
of.boundary.nodal.force.local ‘rock’ x \PYGZhy{}50.0
of.boundary.nodal.force.local ‘rock’ y 27.0
\end{sphinxVerbatim}


\subsection{of.boundary.nodal.velocity.table nodalgroupname xy vel\_x vel\_y tablename tablename}
\label{\detokenize{rst_tutorials/command_line_guide:of-boundary-nodal-velocity-table-nodalgroupname-xy-vel-x-vel-y-tablename-tablename}}

\subsection{of.boundary.nodal.acceleration.table nodalgroupname xy acc\_x acc\_y tablename tablename}
\label{\detokenize{rst_tutorials/command_line_guide:of-boundary-nodal-acceleration-table-nodalgroupname-xy-acc-x-acc-y-tablename-tablename}}

\subsection{of.boundary.pressure.normal nodalgroupname value}
\label{\detokenize{rst_tutorials/command_line_guide:of-boundary-pressure-normal-nodalgroupname-value}}\label{\detokenize{rst_tutorials/command_line_guide:section-10}}

\subsection{of.boundary.pressure.shear nodalgroupname value}
\label{\detokenize{rst_tutorials/command_line_guide:of-boundary-pressure-shear-nodalgroupname-value}}
\sphinxstylestrong{demo}:

\begin{sphinxVerbatim}[commandchars=\\\{\}]
of.boundary.pressure .normal‘rock’ \PYGZhy{}50.0
of.boundary.pressure.shear ‘rock’ \PYGZhy{}50.0
\end{sphinxVerbatim}


\subsection{of.boundary.pressure.normal.table nodalgroupname value}
\label{\detokenize{rst_tutorials/command_line_guide:of-boundary-pressure-normal-table-nodalgroupname-value}}

\subsection{of.boundary.pressure.shear.table nodalgroupname value}
\label{\detokenize{rst_tutorials/command_line_guide:of-boundary-pressure-shear-table-nodalgroupname-value}}
\sphinxstylestrong{demo}:

\begin{sphinxVerbatim}[commandchars=\\\{\}]
of.boundary.pressure.normal.table ‘rock’ \PYGZhy{}50.0 gas
\end{sphinxVerbatim}


\section{hydro}
\label{\detokenize{rst_tutorials/command_line_guide:hydro}}

\subsection{of.boundary.hydro.porepressure nodalgroup p p\_x p\_y}
\label{\detokenize{rst_tutorials/command_line_guide:of-boundary-hydro-porepressure-nodalgroup-p-p-x-p-y}}

\subsection{of.boundary.hydro.waterlevel nodalgroup water\_y initial\_p}
\label{\detokenize{rst_tutorials/command_line_guide:of-boundary-hydro-waterlevel-nodalgroup-water-y-initial-p}}

\subsection{of.boundary.hydro.pressure nodalgroup value}
\label{\detokenize{rst_tutorials/command_line_guide:of-boundary-hydro-pressure-nodalgroup-value}}

\subsection{of.boundary.hydro.pressure.table nodalgroup value tablename}
\label{\detokenize{rst_tutorials/command_line_guide:of-boundary-hydro-pressure-table-nodalgroup-value-tablename}}

\subsection{of.boundary.hydro.flow nodalgroup flow\_rate\_value initial\_p}
\label{\detokenize{rst_tutorials/command_line_guide:of-boundary-hydro-flow-nodalgroup-flow-rate-value-initial-p}}

\subsection{of.boundary.hydro.flow.table nodalgroup value value2 tablename}
\label{\detokenize{rst_tutorials/command_line_guide:of-boundary-hydro-flow-table-nodalgroup-value-value2-tablename}}

\subsection{of.boundary.hydro.impermeable nodalgroup}
\label{\detokenize{rst_tutorials/command_line_guide:of-boundary-hydro-impermeable-nodalgroup}}

\section{blast}
\label{\detokenize{rst_tutorials/command_line_guide:blast}}

\subsection{of.boundary.blast nodalgroup value tablename}
\label{\detokenize{rst_tutorials/command_line_guide:of-boundary-blast-nodalgroup-value-tablename}}

\section{element}
\label{\detokenize{rst_tutorials/command_line_guide:id4}}

\subsection{of.boundary.element.stress elementgroupname sxx sxy syy}
\label{\detokenize{rst_tutorials/command_line_guide:of-boundary-element-stress-elementgroupname-sxx-sxy-syy}}
\sphinxstylestrong{demo}:

\begin{sphinxVerbatim}[commandchars=\\\{\}]
of.boundary.element.stress ‘all’ \PYGZhy{}20e6 2e6 \PYGZhy{}2e6
of.boundary.element.stress.xgrad elementgroupname sxx.x sxy.x syy.x
of.boundary.element.stress.ygrad elementgroupname sxx.y sxy.y syy.y
\end{sphinxVerbatim}


\subsection{of.boundary.excavation elementgroupname}
\label{\detokenize{rst_tutorials/command_line_guide:of-boundary-excavation-elementgroupname}}

\subsection{of.boundary.uncontact elementgroupname}
\label{\detokenize{rst_tutorials/command_line_guide:of-boundary-uncontact-elementgroupname}}

\section{thermal}
\label{\detokenize{rst_tutorials/command_line_guide:thermal}}

\subsection{of.boundary.thermal.t0 nodalgroupvalue}
\label{\detokenize{rst_tutorials/command_line_guide:of-boundary-thermal-t0-nodalgroupvalue}}

\subsection{of.boundary.thermal.temperature nodalgroupvalue}
\label{\detokenize{rst_tutorials/command_line_guide:of-boundary-thermal-temperature-nodalgroupvalue}}

\chapter{material}
\label{\detokenize{rst_tutorials/command_line_guide:material}}\label{\detokenize{rst_tutorials/command_line_guide:section-15}}

\section{of.mat.element elementgroupname(all) modelname p1 p2 p3 …}
\label{\detokenize{rst_tutorials/command_line_guide:of-mat-element-elementgroupname-all-modelname-p1-p2-p3}}
\sphinxstylestrong{demo}:

\begin{sphinxVerbatim}[commandchars=\\\{\}]
of.mat.element ‘all’ elastic den 2500 e 40e9 v 0.2
of.mat.element ‘rock’ mc den 2500 e 40e9 v 0.2 ten 10e6 coh 30e6 fric 0.3
\end{sphinxVerbatim}


\section{of.mat.particle elementgroupname(all) modelname p1 p2 p3 …}
\label{\detokenize{rst_tutorials/command_line_guide:of-mat-particle-elementgroupname-all-modelname-p1-p2-p3}}
\sphinxstylestrong{demo}:

\begin{sphinxVerbatim}[commandchars=\\\{\}]
of.mat.particle ‘all’ rigid den 2500
\end{sphinxVerbatim}


\section{of.mat.cohesive cohelementgroupname(all) modelname p1 p2 p3 …}
\label{\detokenize{rst_tutorials/command_line_guide:of-mat-cohesive-cohelementgroupname-all-modelname-p1-p2-p3}}
\sphinxstylestrong{demo}:

\begin{sphinxVerbatim}[commandchars=\\\{\}]
of.mat.cohesive ‘all’ em pn 1e10 pt 0.5e10 ten 30e6 coh 30e6 fric 0.3 gi 100 gii 300 (pn pt ten coh fri gi gii)
\end{sphinxVerbatim}


\section{of.mat. cohesive ‘rock’ em\_het power 0.5 dip 35 ten 30e6 3e6 coh 30e6 3e6 fric 0.3 0 gi 100 20 gii 300 30 (power, dip, mean pn, dev pn, mean pt, dev pt, mean ten, dev ten, mean coh, dev coh, mean fri, dev fri, mean gi, dev gi, mean gii, dev gii)}
\label{\detokenize{rst_tutorials/command_line_guide:of-mat-cohesive-rock-em-het-power-0-5-dip-35-ten-30e6-3e6-coh-30e6-3e6-fric-0-3-0-gi-100-20-gii-300-30-power-dip-mean-pn-dev-pn-mean-pt-dev-pt-mean-ten-dev-ten-mean-coh-dev-coh-mean-fri-dev-fri-mean-gi-dev-gi-mean-gii-dev-gii}}

\section{of.mat.contact elementgroupname1 elementgroupname2(all) modelname p1 p2 p3 …}
\label{\detokenize{rst_tutorials/command_line_guide:of-mat-contact-elementgroupname1-elementgroupname2-all-modelname-p1-p2-p3}}\label{\detokenize{rst_tutorials/command_line_guide:section-16}}
\sphinxstylestrong{demo}:

\begin{sphinxVerbatim}[commandchars=\\\{\}]
of.mat.contact ‘all’ mc kn 2e10 ks 1e10 fric 0.3
of.mat.contact ‘rock’ ‘plate’ mc kn 2e10 ks 1e10 fric 0.3 kn ks
\end{sphinxVerbatim}


\subsection{of.mat.fluid den bulk viscousity cohesion}
\label{\detokenize{rst_tutorials/command_line_guide:of-mat-fluid-den-bulk-viscousity-cohesion}}
\sphinxstylestrong{demo}:

\begin{sphinxVerbatim}[commandchars=\\\{\}]
\PYG{n}{of}\PYG{o}{.}\PYG{n}{mat}\PYG{o}{.}\PYG{n}{fluid} \PYG{n}{den} \PYG{l+m+mf}{1000.0} \PYG{n}{k} \PYG{l+m+mf}{3e8} \PYG{n}{viscosity} \PYG{l+m+mf}{1e6} \PYG{n}{cohesion} \PYG{l+m+mf}{3e6}
\end{sphinxVerbatim}


\subsection{of.mat.fluid.matrix elementgroupname(all) permeability m alpha}
\label{\detokenize{rst_tutorials/command_line_guide:of-mat-fluid-matrix-elementgroupname-all-permeability-m-alpha}}
\sphinxstylestrong{demo}:

\begin{sphinxVerbatim}[commandchars=\\\{\}]
\PYG{n}{of}\PYG{o}{.}\PYG{n}{mat}\PYG{o}{.}\PYG{n}{fluid}\PYG{o}{.}\PYG{n}{matrix} \PYG{l+s+s1}{\PYGZsq{}}\PYG{l+s+s1}{all}\PYG{l+s+s1}{\PYGZsq{}} \PYG{n}{permeability} \PYG{l+m+mf}{1.0e\PYGZhy{}8} \PYG{n}{biot\PYGZus{}k} \PYG{l+m+mf}{22e9} \PYG{n}{biot\PYGZus{}c} \PYG{l+m+mf}{0.1}
\end{sphinxVerbatim}


\subsection{of.mat.fluid.fracture cohelementgroupname(all) aperature\_0 apreature\_min para\_exp para\_b}
\label{\detokenize{rst_tutorials/command_line_guide:of-mat-fluid-fracture-cohelementgroupname-all-aperature-0-apreature-min-para-exp-para-b}}
\sphinxstylestrong{demo}:

\begin{sphinxVerbatim}[commandchars=\\\{\}]
\PYG{n}{of}\PYG{o}{.}\PYG{n}{mat}\PYG{o}{.}\PYG{n}{fluid}\PYG{o}{.}\PYG{n}{fracture} \PYG{l+s+s1}{\PYGZsq{}}\PYG{l+s+s1}{joint}\PYG{l+s+s1}{\PYGZsq{}} \PYG{n}{a0} \PYG{l+m+mf}{5e\PYGZhy{}4} \PYG{n}{power} \PYG{l+m+mf}{3.0} \PYG{n}{b} \PYG{l+m+mf}{1.0}
\end{sphinxVerbatim}


\subsection{of.mat.gas initial\_den permeability initial\_bulk constant\_b alpha}
\label{\detokenize{rst_tutorials/command_line_guide:of-mat-gas-initial-den-permeability-initial-bulk-constant-b-alpha}}
\sphinxstylestrong{demo}:

\begin{sphinxVerbatim}[commandchars=\\\{\}]
\PYG{n}{of}\PYG{o}{.}\PYG{n}{mat}\PYG{o}{.}\PYG{n}{fluid}\PYG{o}{.}\PYG{n}{fracture} \PYG{l+s+s1}{\PYGZsq{}}\PYG{l+s+s1}{joint}\PYG{l+s+s1}{\PYGZsq{}} \PYG{n}{density0} \PYG{l+m+mf}{1.29} \PYG{n}{permiability} \PYG{l+m+mi}{200} \PYG{n}{k0} \PYG{l+m+mf}{1.01e5} \PYG{n}{b}
\PYG{l+m+mf}{1.0} \PYG{n}{alpha} \PYG{l+m+mf}{0.1}
\end{sphinxVerbatim}


\chapter{gbm}
\label{\detokenize{rst_tutorials/command_line_guide:gbm}}

\section{of.gbm numberofminerals m1\_name ratio m2\_name ratio …}
\label{\detokenize{rst_tutorials/command_line_guide:of-gbm-numberofminerals-m1-name-ratio-m2-name-ratio}}
\sphinxstylestrong{demo}:

\begin{sphinxVerbatim}[commandchars=\\\{\}]
\PYG{n}{of}\PYG{o}{.}\PYG{n}{gbm} \PYG{l+m+mi}{3} \PYG{l+s+s1}{\PYGZsq{}}\PYG{l+s+s1}{qtz}\PYG{l+s+s1}{\PYGZsq{}} \PYG{l+m+mf}{0.3} \PYG{l+s+s1}{\PYGZsq{}}\PYG{l+s+s1}{fel}\PYG{l+s+s1}{\PYGZsq{}} \PYG{l+m+mf}{0.46} \PYG{l+s+s1}{\PYGZsq{}}\PYG{l+s+s1}{bio}\PYG{l+s+s1}{\PYGZsq{}} \PYG{l+m+mf}{0.24} \PYG{p}{(}\PYG{n}{area} \PYG{n}{ratio}\PYG{p}{)}
\end{sphinxVerbatim}


\chapter{history}
\label{\detokenize{rst_tutorials/command_line_guide:history}}

\section{of.history.nodal.force id x1 y1}
\label{\detokenize{rst_tutorials/command_line_guide:of-history-nodal-force-id-x1-y1}}

\section{of.history.nodal.vel id x1 y1}
\label{\detokenize{rst_tutorials/command_line_guide:of-history-nodal-vel-id-x1-y1}}

\section{of.history.nodal.dis id x1 y1}
\label{\detokenize{rst_tutorials/command_line_guide:of-history-nodal-dis-id-x1-y1}}

\section{of.history.nodal.fluid.pressure id x1 y1}
\label{\detokenize{rst_tutorials/command_line_guide:of-history-nodal-fluid-pressure-id-x1-y1}}

\section{of.history.nodal.fracture.pressure id x1 y1}
\label{\detokenize{rst_tutorials/command_line_guide:of-history-nodal-fracture-pressure-id-x1-y1}}

\section{of.history.nodal.matrix.pressure id x1 y1}
\label{\detokenize{rst_tutorials/command_line_guide:of-history-nodal-matrix-pressure-id-x1-y1}}

\section{of.history.nodal.temperature id x1 y1}
\label{\detokenize{rst_tutorials/command_line_guide:of-history-nodal-temperature-id-x1-y1}}

\section{of.history.nodal.group.force id groupname}
\label{\detokenize{rst_tutorials/command_line_guide:of-history-nodal-group-force-id-groupname}}\label{\detokenize{rst_tutorials/command_line_guide:section-17}}

\section{of.history.nodal.group.vel id groupname}
\label{\detokenize{rst_tutorials/command_line_guide:of-history-nodal-group-vel-id-groupname}}

\section{of.history.nodal.group.dis id groupname}
\label{\detokenize{rst_tutorials/command_line_guide:of-history-nodal-group-dis-id-groupname}}

\section{of.history.nodal.group.fluid.pressure id groupname}
\label{\detokenize{rst_tutorials/command_line_guide:of-history-nodal-group-fluid-pressure-id-groupname}}

\section{of.history.nodal.group.fracture.pressure id groupname}
\label{\detokenize{rst_tutorials/command_line_guide:of-history-nodal-group-fracture-pressure-id-groupname}}

\section{of.history.nodal.group.matrix.pressure id groupname}
\label{\detokenize{rst_tutorials/command_line_guide:of-history-nodal-group-matrix-pressure-id-groupname}}

\section{of.history.nodal.group.temperature id groupname}
\label{\detokenize{rst_tutorials/command_line_guide:of-history-nodal-group-temperature-id-groupname}}\begin{quote}

note: average value in this group
\end{quote}


\section{of.history.element.stress id x1 y1}
\label{\detokenize{rst_tutorials/command_line_guide:of-history-element-stress-id-x1-y1}}

\section{of.history.element.strain id x1 y1}
\label{\detokenize{rst_tutorials/command_line_guide:of-history-element-strain-id-x1-y1}}

\section{of.history.element.strainrate id x1 y1}
\label{\detokenize{rst_tutorials/command_line_guide:of-history-element-strainrate-id-x1-y1}}

\section{of.history.element.group.stress id groupname}
\label{\detokenize{rst_tutorials/command_line_guide:of-history-element-group-stress-id-groupname}}\label{\detokenize{rst_tutorials/command_line_guide:section-18}}

\section{of.history.element.group.strain id groupname}
\label{\detokenize{rst_tutorials/command_line_guide:of-history-element-group-strain-id-groupname}}

\section{of.history.element.group.strainrate id groupname}
\label{\detokenize{rst_tutorials/command_line_guide:of-history-element-group-strainrate-id-groupname}}

\section{of.history.cohelement.dis id x1 y1}
\label{\detokenize{rst_tutorials/command_line_guide:of-history-cohelement-dis-id-x1-y1}}\label{\detokenize{rst_tutorials/command_line_guide:section-19}}

\section{of.history.cohelement.force id x1 y1}
\label{\detokenize{rst_tutorials/command_line_guide:of-history-cohelement-force-id-x1-y1}}

\section{of.history.cohelement.vel id x1 y1}
\label{\detokenize{rst_tutorials/command_line_guide:of-history-cohelement-vel-id-x1-y1}}

\section{of.history.cohelement.shearstrength id x1 y1}
\label{\detokenize{rst_tutorials/command_line_guide:of-history-cohelement-shearstrength-id-x1-y1}}

\section{of.history.cohelement.group.dis id groupname}
\label{\detokenize{rst_tutorials/command_line_guide:of-history-cohelement-group-dis-id-groupname}}\label{\detokenize{rst_tutorials/command_line_guide:section-20}}

\section{of.history.cohelement.group.force id groupname}
\label{\detokenize{rst_tutorials/command_line_guide:of-history-cohelement-group-force-id-groupname}}

\section{of.history.cohelement.group.vel id groupname}
\label{\detokenize{rst_tutorials/command_line_guide:of-history-cohelement-group-vel-id-groupname}}

\section{of.history.cohelement.group.shearstrength id groupname}
\label{\detokenize{rst_tutorials/command_line_guide:of-history-cohelement-group-shearstrength-id-groupname}}

\section{of.history.energy}
\label{\detokenize{rst_tutorials/command_line_guide:of-history-energy}}\label{\detokenize{rst_tutorials/command_line_guide:section-22}}\label{\detokenize{rst_tutorials/command_line_guide:section-21}}

\section{of.history.unbalance}
\label{\detokenize{rst_tutorials/command_line_guide:of-history-unbalance}}

\section{of.history.solveratio}
\label{\detokenize{rst_tutorials/command_line_guide:of-history-solveratio}}

\section{of.history.interval intervalvalue}
\label{\detokenize{rst_tutorials/command_line_guide:of-history-interval-intervalvalue}}\label{\detokenize{rst_tutorials/command_line_guide:section-23}}

\section{of.history.pv.interval intervalvalue}
\label{\detokenize{rst_tutorials/command_line_guide:of-history-pv-interval-intervalvalue}}

\section{of.history.pv.reduced.interval intervalvalue fracturethreshold intervalvalue}
\label{\detokenize{rst_tutorials/command_line_guide:of-history-pv-reduced-interval-intervalvalue-fracturethreshold-intervalvalue}}

\section{paraview}
\label{\detokenize{rst_tutorials/command_line_guide:paraview}}

\subsection{of.history.pv.field fieldkeywords}
\label{\detokenize{rst_tutorials/command_line_guide:of-history-pv-field-fieldkeywords}}\begin{quote}

fieldkeywords: velocity force displacement fluid\_pressure nodal\_group
element\_group gbm\_group mass stress strain strain\_rate
principal\_stress mat\_id fragment
\end{quote}


\subsection{of.history.pv.fracture fracturekeywords}
\label{\detokenize{rst_tutorials/command_line_guide:of-history-pv-fracture-fracturekeywords}}\begin{quote}

fracturekeywords: mode sliding opening area time length energy
\end{quote}


\subsection{of.history.pv.damage damagekeywords}
\label{\detokenize{rst_tutorials/command_line_guide:of-history-pv-damage-damagekeywords}}\begin{quote}

damagekeywords: mode sliding opening area time length
\end{quote}


\subsection{of.history.pv.cohesive cohesivekeywords}
\label{\detokenize{rst_tutorials/command_line_guide:of-history-pv-cohesive-cohesivekeywords}}\begin{quote}

cohesivekeywords: velocity force displacement shear\_strength dfn
mat\_id group
\end{quote}


\subsection{of.history.pv.ae aekeywords}
\label{\detokenize{rst_tutorials/command_line_guide:of-history-pv-ae-aekeywords}}\begin{quote}

aekeywords: mode time win\_time win\_kinetic kinetic magnitude energy
\end{quote}


\chapter{dfn}
\label{\detokenize{rst_tutorials/command_line_guide:dfn}}

\section{of.dfn.connectivity dfnnum}
\label{\detokenize{rst_tutorials/command_line_guide:of-dfn-connectivity-dfnnum}}\begin{quote}

n1 n2 …
\end{quote}

\sphinxstylestrong{demo}:

\begin{sphinxVerbatim}[commandchars=\\\{\}]
of.dfn.connectivity 1000 1 15 2 19 1000 999 …
of.dfn.group dfnname setnum dfn1 dfn4 dfn12…
\end{sphinxVerbatim}

\sphinxstylestrong{demo}:

\begin{sphinxVerbatim}[commandchars=\\\{\}]
of.dfn.group ‘dfn\PYGZus{}1’ 3 1 7 12
\end{sphinxVerbatim}


\subsection{of.dfn.type dfnname dfn\_enum}
\label{\detokenize{rst_tutorials/command_line_guide:of-dfn-type-dfnname-dfn-enum}}\label{\detokenize{rst_tutorials/command_line_guide:section-24}}
\sphinxstylestrong{demo}:

\begin{sphinxVerbatim}[commandchars=\\\{\}]
of.dfn.type ‘jset1’ cohesive
\end{sphinxVerbatim}


\chapter{table}
\label{\detokenize{rst_tutorials/command_line_guide:table}}

\section{of.table tablename ‘table1.dat’}
\label{\detokenize{rst_tutorials/command_line_guide:of-table-tablename-table1-dat}}
format of table:

\begin{sphinxVerbatim}[commandchars=\\\{\}]
t1 t2 num
n1 n2 n3 …
struct table:
\PYGZob{}char\PYGZbs{}* tag;
double t1;
double t2;
unsigned int num;
double\PYGZbs{}* data;\PYGZcb{}
\end{sphinxVerbatim}

\sphinxstylestrong{demo}:

\begin{sphinxVerbatim}[commandchars=\\\{\}]
of.table tab1 ‘table1.dat’
\end{sphinxVerbatim}


\chapter{damping}
\label{\detokenize{rst_tutorials/command_line_guide:damping}}

\section{of.damp.global value}
\label{\detokenize{rst_tutorials/command_line_guide:of-damp-global-value}}
\sphinxstylestrong{demo}:

\begin{sphinxVerbatim}[commandchars=\\\{\}]
\PYG{n}{of}\PYG{o}{.}\PYG{n}{damp}\PYG{o}{.}\PYG{k}{global} \PYG{l+m+mf}{0.7}
\end{sphinxVerbatim}

\sphinxstylestrong{demo}:

\begin{sphinxVerbatim}[commandchars=\\\{\}]
\PYG{n}{of}\PYG{o}{.}\PYG{n}{damp}\PYG{o}{.}\PYG{n}{rayleigh} \PYG{n}{elementgroupname} \PYG{n}{value1} \PYG{n}{value2}
\PYG{n}{of}\PYG{o}{.}\PYG{n}{damp}\PYG{o}{.}\PYG{n}{rayleigh}\PYG{o}{.}\PYG{n}{mass} \PYG{n}{elementgroupname} \PYG{n}{value}
\PYG{n}{of}\PYG{o}{.}\PYG{n}{damp}\PYG{o}{.}\PYG{n}{rayleigh}\PYG{o}{.}\PYG{n}{stiffness} \PYG{n}{elementgroupname} \PYG{n}{value}
\end{sphinxVerbatim}


\chapter{hydro}
\label{\detokenize{rst_tutorials/command_line_guide:id5}}

\section{of.hydro.timestep value}
\label{\detokenize{rst_tutorials/command_line_guide:of-hydro-timestep-value}}
\sphinxstylestrong{demo}:

\begin{sphinxVerbatim}[commandchars=\\\{\}]
\PYG{n}{of}\PYG{o}{.}\PYG{n}{hydro}\PYG{o}{.}\PYG{n}{timestep} \PYG{l+m+mf}{1e\PYGZhy{}9}
\PYG{n}{of}\PYG{o}{.}\PYG{n}{hydro}\PYG{o}{.}\PYG{n}{matrix} \PYG{n}{off}
\PYG{n}{of}\PYG{o}{.}\PYG{n}{hydro}\PYG{o}{.}\PYG{n}{mechanical} \PYG{n}{off}
\PYG{n}{of}\PYG{o}{.}\PYG{n}{hydro}\PYG{o}{.}\PYG{n}{fracture} \PYG{n}{on}
\end{sphinxVerbatim}


\chapter{gravity}
\label{\detokenize{rst_tutorials/command_line_guide:gravity}}

\section{of.gravity x y}
\label{\detokenize{rst_tutorials/command_line_guide:of-gravity-x-y}}\begin{description}
\item[{\sphinxstylestrong{demo}::}] \leavevmode
of.gravity 0.0 \sphinxhyphen{}9.8

\end{description}


\chapter{seismic}
\label{\detokenize{rst_tutorials/command_line_guide:seismic}}

\section{of.seismic.window value}
\label{\detokenize{rst_tutorials/command_line_guide:of-seismic-window-value}}
\sphinxstylestrong{demo}:

\begin{sphinxVerbatim}[commandchars=\\\{\}]
\PYG{n}{of}\PYG{o}{.}\PYG{n}{seismic}\PYG{o}{.}\PYG{n}{window} \PYG{l+m+mf}{1.0e\PYGZhy{}3}
\end{sphinxVerbatim}
\begin{enumerate}
\sphinxsetlistlabels{\arabic}{enumi}{enumii}{}{.}%
\item {} 
of.timestep auto or fix value

\item {} 
of.step

\item {} 
of.stop

\end{enumerate}



\renewcommand{\indexname}{Index}
\printindex
\end{document}